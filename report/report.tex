% !TEX TS-program = LuaLaTeX
\documentclass[10pt, oneside, a4paper]{article}

\usepackage[T1]{fontenc}
\usepackage{lmodern}
\usepackage{xcolor}
    \definecolor{gray} {HTML}{363636}
    \definecolor{red}  {HTML}{950009}
    \definecolor{green}{HTML}{0E610A}
    \definecolor{blue} {HTML}{020069}
\usepackage{fontspec}
    \setsansfont{Arial}
\usepackage{amsmath}
\usepackage{titlesec}
    \titleformat*{\section}      {\color{gray}\large\bfseries\sffamily}
    \titleformat*{\subsection}   {\color{gray}\large\bfseries\sffamily}
    \titleformat*{\subsubsection}{\color{gray}\large\bfseries\sffamily}
\usepackage{geometry}
%   \geometry{showframe}
    \geometry{scale={0.75,0.85}}
%\usepackage{pgfplots}
%    \pgfplotsset{compat=newest}
\usepackage{siunitx}
    \sisetup{locale=FR}
\usepackage{graphicx}
\usepackage{caption}
    \captionsetup{labelfont={bf,sf,color=gray}}
\usepackage{pdfpages}
\usepackage{pgfplots}
	\pgfplotsset{compat=newest}
	\pgfplotsset{plot coordinates/math parser=false}
	\newlength\figureheight
	\newlength\figurewidth

% Keep lasts
\usepackage[french]{babel}
	\frenchsetup{SmallCapsFigTabCaptions=false}
\usepackage[expansion]{microtype}
\usepackage[luatex, backref]{hyperref}
    \hypersetup{unicode, colorlinks, breaklinks, urlcolor=red,
                bookmarksopen, bookmarksnumbered}

\renewcommand{\UrlFont}{\small}
\renewcommand{\arraystretch}{1.1}
\newcommand{\important}[1]{\textbf{\textsf{\color{gray}{#1}}}}
\setlength{\parskip}{2mm}

\begin{titlepage}
    \date{%
        \today
    }
    \author{%
        Alexis~\bsc{Nootens} \\
        \href{mailto:16139@student.ecam.be}{16139@student.ecam.be}
        \and
        Thomas~\bsc{Anizet} \\
        \href{mailto:14164@student.ecam.be}{14164@student.ecam.be}
    }
    \title{%
        \color{gray}\LARGE\bfseries\sffamily
        Projet MATLAB\textregistered{} 2018             \\[3mm]
        \rm\sffamily\large
        Simulation d'une ligne de transmission physique \\[7mm]
        École Centrale des Arts et Métiers 		        \\
        1200 Woluwe-Saint-Lambert				        \\
        Belgique
    }
\end{titlepage}

\begin{document}
\maketitle

\section*{Introduction}

\begin{figure}[htbp]
	\centering
	\resizebox{\textwidth}{!}{%
		% !TEX root = ../report.tex
% This file was created by matlab2tikz.
%
%The latest updates can be retrieved from
%  http://www.mathworks.com/matlabcentral/fileexchange/22022-matlab2tikz-matlab2tikz
%where you can also make suggestions and rate matlab2tikz.
%
\definecolor{mycolor1}{rgb}{0.00000,0.44700,0.74100}%
%
\begin{tikzpicture}

\begin{axis}[%
width=10.242in,
height=8.535in,
at={(5.229in,1.152in)},
scale only axis,
xmin=0,
xmax=1200,
ymin=-0.5,
ymax=0.5,
axis background/.style={fill=white},
xmajorgrids,
ymajorgrids,
legend style={legend cell align=left, align=left, draw=white!15!black}
]
\addplot[ycomb, color=mycolor1, mark=o, mark options={solid, mycolor1}] table[row sep=crcr] {%
1	-0.000576556801538217\\
2	-0.00054160203012496\\
3	-0.000431459154985877\\
4	-0.000257748625914949\\
5	-4.10576122895505e-05\\
6	0.000191499263699676\\
7	0.000409679056841888\\
8	0.00058404841971295\\
9	0.000689952691577969\\
10	0.000710986583408642\\
11	0.000641484689394609\\
12	0.000487663689891892\\
13	0.000843771599140943\\
14	0.000548946511467299\\
15	0.000172927619176001\\
16	-0.00023730014621486\\
17	-0.000628517196327915\\
18	-0.000948199394793136\\
19	-0.00115160994730343\\
20	-0.00120810707676138\\
21	-0.00110579111380619\\
22	-0.000853803929891216\\
23	-0.000481881867820496\\
24	-3.711582958498e-05\\
25	-0.00015480127167794\\
26	0.000290248260636651\\
27	0.000703049659875189\\
28	0.00102568211114391\\
29	0.0012104225062538\\
30	0.00122668778371679\\
31	0.00106593280022171\\
32	0.000743753215867917\\
33	0.000298788699481878\\
34	-0.000211563240673284\\
35	-0.000718181051957134\\
36	-0.00114916808383463\\
37	-0.000863031572387148\\
38	-0.000999101820665712\\
39	-0.000995977218912057\\
40	-0.000845210473686643\\
41	-0.000558600900573898\\
42	-0.000167662380451185\\
43	0.000279669158379722\\
44	0.000724783754505198\\
45	0.00110594233061237\\
46	0.0013666770423095\\
47	0.00146393572069829\\
48	0.00137483151177671\\
49	0.000524428447552996\\
50	0.00012811803550976\\
51	-0.000299702473153716\\
52	-0.000702230913732165\\
53	-0.00102243561801298\\
54	-0.00121092873035368\\
55	-0.00123337234944321\\
56	-0.00107637310006176\\
57	-0.000750946442714807\\
58	-0.000292890295737177\\
59	0.000240226080601655\\
60	0.000775318009308614\\
61	0.0018092538443874\\
62	0.00207824530725089\\
63	0.00205779682469644\\
64	0.00172344548176973\\
65	0.0010917352034456\\
66	0.000221336358999349\\
67	-0.000791357287318945\\
68	-0.00182200942483781\\
69	-0.00273264338876744\\
70	-0.00338895300536274\\
71	-0.00367859600263716\\
72	-0.00352814932164039\\
73	-0.00349294098202525\\
74	-0.00242339671269136\\
75	-0.000954335497546489\\
76	0.000751798378879301\\
77	0.00248663065866343\\
78	0.00402007992942013\\
79	0.00512974026025942\\
80	0.00563203082237963\\
81	0.00541118156595409\\
82	0.004442034757998\\
83	0.00280305012297506\\
84	0.000676798206199617\\
85	-0.00108689878494887\\
86	-0.0033392569402395\\
87	-0.00518235375743825\\
88	-0.00626516316392185\\
89	-0.00629989765093389\\
90	-0.00511157276304274\\
91	-0.00267937385372199\\
92	0.000836469645950601\\
93	0.00508755807007867\\
94	0.00955342969634078\\
95	0.013582027602221\\
96	0.0164529121191657\\
97	0.0168822134065909\\
98	0.0154565483171755\\
99	0.0112387949568401\\
100	0.00410357430324947\\
101	-0.00572917379934072\\
102	-0.0176535201753458\\
103	-0.0306727979627754\\
104	-0.0434360309963856\\
105	-0.054317051409225\\
106	-0.0615335507699705\\
107	-0.0632989150343361\\
108	-0.0579956170946642\\
109	-0.0449321817919581\\
110	-0.0221727279638924\\
111	0.00983036758157343\\
112	0.050385435943171\\
113	0.0979850518065775\\
114	0.150332668919082\\
115	0.204441463608494\\
116	0.25680302033635\\
117	0.303616939833885\\
118	0.341066291299018\\
119	0.365618665283382\\
120	0.374328946937716\\
121	0.364541676957409\\
122	0.336462215329596\\
123	0.289826431907092\\
124	0.22621973662297\\
125	0.148513666393112\\
126	0.0607358218466689\\
127	-0.0321566597436985\\
128	-0.124586890594684\\
129	-0.21073071806337\\
130	-0.284925445326916\\
131	-0.342084409413059\\
132	-0.378085499423786\\
133	-0.390679275092645\\
134	-0.377395045858452\\
135	-0.339171346162696\\
136	-0.278128688562831\\
137	-0.198000613954221\\
138	-0.103908808154044\\
139	-0.00203137003420167\\
140	0.100813069459838\\
141	0.197635870323146\\
142	0.281769460598966\\
143	0.347342899794843\\
144	0.38970664607428\\
145	0.405198180673212\\
146	0.393716923510401\\
147	0.355342541946329\\
148	0.292556265425853\\
149	0.209545542579651\\
150	0.111921934471965\\
151	0.00633708239653918\\
152	-0.0999762165491719\\
153	-0.199705451545177\\
154	-0.285966993648013\\
155	-0.352786030529963\\
156	-0.395514117597769\\
157	-0.411731635730929\\
158	-0.399118074869569\\
159	-0.359023329151685\\
160	-0.294165954039372\\
161	-0.208982888450241\\
162	-0.109324469446442\\
163	-0.00205075984992562\\
164	0.105442799890767\\
165	0.20573749495332\\
166	0.291905642737803\\
167	0.357991373979578\\
168	0.399424447363941\\
169	0.412761697886429\\
170	0.398228169999032\\
171	0.356295906607041\\
172	0.289863769508193\\
173	0.203522013741848\\
174	0.103234345806815\\
175	-0.00407548948668777\\
176	-0.111003990258745\\
177	-0.210181697484818\\
178	-0.294784709542825\\
179	-0.359008137349641\\
180	-0.398468265474231\\
181	-0.411081640576627\\
182	-0.394905396847033\\
183	-0.35167387477405\\
184	-0.284459836270402\\
185	-0.197987571006643\\
186	-0.0982948436902603\\
187	0.00769703254165398\\
188	0.1126738439464\\
189	0.209444235203674\\
190	0.291443150287617\\
191	0.353186508176645\\
192	0.390644398005143\\
193	0.402082846538782\\
194	0.385861580660277\\
195	0.343927039274484\\
196	0.279433864245001\\
197	0.197027330124083\\
198	0.1024884879281\\
199	0.00230519036236822\\
200	-0.0967974696474398\\
201	-0.188331265471641\\
202	-0.266487840776088\\
203	-0.32651819914501\\
204	-0.365022041978157\\
205	-0.379551557044703\\
206	-0.371014246336958\\
207	-0.340126828582174\\
208	-0.289497900167858\\
209	-0.222836182509797\\
210	-0.144627518663516\\
211	-0.0597740865297846\\
212	0.0267742047058587\\
213	0.110367103690928\\
214	0.186954479993977\\
215	0.253319463128565\\
216	0.307230613285882\\
217	0.346930456292521\\
218	0.373456178693927\\
219	0.387053056916152\\
220	0.389263265428156\\
221	0.382187704352614\\
222	0.368260505576358\\
223	0.350022125871828\\
224	0.32991004870893\\
225	0.310082807549084\\
226	0.292288605412225\\
227	0.277784743355838\\
228	0.267308912143627\\
229	0.260522086797414\\
230	0.258409676897449\\
231	0.25988253420454\\
232	0.264134136485632\\
233	0.270220856013992\\
234	0.277172944095156\\
235	0.284091785191309\\
236	0.290226317839696\\
237	0.295024378658047\\
238	0.298157640348401\\
239	0.29952148873017\\
240	0.299213379672833\\
241	0.296918217316728\\
242	0.294735858250301\\
243	0.291389903351216\\
244	0.287912642583009\\
245	0.284721855811174\\
246	0.282160042386514\\
247	0.280469089571829\\
248	0.279777643282936\\
249	0.280100586751736\\
250	0.281349006186284\\
251	0.283348331773139\\
252	0.285861998923344\\
253	0.289194503077896\\
254	0.29134222491737\\
255	0.293756878085978\\
256	0.295630723489675\\
257	0.296791378208156\\
258	0.297138341053114\\
259	0.296649875169391\\
260	0.295383545292896\\
261	0.29347058440906\\
262	0.291104543982697\\
263	0.28852494058081\\
264	0.285996867598725\\
265	0.283211245941171\\
266	0.282136368491724\\
267	0.281255548803921\\
268	0.281282047275243\\
269	0.282264321061885\\
270	0.284150706194576\\
271	0.286786208161352\\
272	0.289919210678221\\
273	0.293218552464985\\
274	0.296300452793893\\
275	0.298763713327768\\
276	0.300230591145449\\
277	0.300966400291131\\
278	0.299044541980451\\
279	0.296119468858428\\
280	0.291714061167641\\
281	0.286106716319354\\
282	0.279748385351556\\
283	0.273237405775423\\
284	0.267276080799318\\
285	0.262611444894965\\
286	0.259964501936898\\
287	0.259953775190912\\
288	0.263020080108744\\
289	0.268783316812451\\
290	0.278867523080583\\
291	0.291133334993145\\
292	0.305411501642276\\
293	0.320650677406494\\
294	0.335546296075084\\
295	0.348618515354299\\
296	0.358310354025359\\
297	0.363098421465259\\
298	0.361607132557436\\
299	0.352716632416597\\
300	0.335654921163975\\
301	0.310642426803204\\
302	0.276053594708966\\
303	0.234158837628798\\
304	0.185357092421808\\
305	0.130990605593311\\
306	0.0726803738676083\\
307	0.0122192434026343\\
308	-0.0485443082715009\\
309	-0.107821327447746\\
310	-0.163986328444585\\
311	-0.215661126714977\\
312	-0.261771802609176\\
313	-0.300998298929209\\
314	-0.333575345677121\\
315	-0.360018280787973\\
316	-0.379826621595489\\
317	-0.39322044080017\\
318	-0.400489482409167\\
319	-0.401923578728082\\
320	-0.397756208184381\\
321	-0.388127953587948\\
322	-0.373074253036866\\
323	-0.352538985411141\\
324	-0.326412397100807\\
325	-0.294012416965159\\
326	-0.257031659116826\\
327	-0.213874559091442\\
328	-0.165464387750241\\
329	-0.112421789199761\\
330	-0.0556737235054509\\
331	0.00354247745675339\\
332	0.0637069137296358\\
333	0.123077259266734\\
334	0.179778500500895\\
335	0.231916439110288\\
336	0.277706456258378\\
337	0.315030485045301\\
338	0.344439738457936\\
339	0.363524537083833\\
340	0.372729682919268\\
341	0.372517532675963\\
342	0.363939829045017\\
343	0.348586599928432\\
344	0.328489850945344\\
345	0.305986991558942\\
346	0.283552485322963\\
347	0.263609255595289\\
348	0.248333651700428\\
349	0.240045634544542\\
350	0.239240030767736\\
351	0.245702073741423\\
352	0.259671949315183\\
353	0.28001998465047\\
354	0.304866629871071\\
355	0.331688528063741\\
356	0.357483112821966\\
357	0.378982752863788\\
358	0.39290495947586\\
359	0.396221410231428\\
360	0.386425876771786\\
361	0.361203315002404\\
362	0.321508422300669\\
363	0.265967901518854\\
364	0.196680524485811\\
365	0.11631602577172\\
366	0.0285762881023061\\
367	-0.0620068805399501\\
368	-0.15035028246456\\
369	-0.231156978908683\\
370	-0.299290181384574\\
371	-0.35015636305593\\
372	-0.380068368521749\\
373	-0.385982794673382\\
374	-0.367544325962039\\
375	-0.32598605188771\\
376	-0.262917865747435\\
377	-0.182043193878435\\
378	-0.0883769321081098\\
379	0.0120649406050059\\
380	0.112660865253749\\
381	0.206637079729295\\
382	0.287546070474807\\
383	0.349734212541587\\
384	0.388762448193315\\
385	0.402323094052233\\
386	0.388665380798885\\
387	0.347938648111147\\
388	0.283395569195418\\
389	0.199408765768626\\
390	0.101741779972979\\
391	-0.0028625582113663\\
392	-0.107168635664197\\
393	-0.203972065248971\\
394	-0.286620710484876\\
395	-0.349492473612008\\
396	-0.388393178817342\\
397	-0.401421060750715\\
398	-0.386247728255859\\
399	-0.345872295102797\\
400	-0.282760642028721\\
401	-0.201471150721148\\
402	-0.107722043290519\\
403	-0.00795947967875195\\
404	0.0911149045393608\\
405	0.183019609308898\\
406	0.261935631550262\\
407	0.323083189371659\\
408	0.363006980574013\\
409	0.380329142375317\\
410	0.37400259488777\\
411	0.344495515632734\\
412	0.294870379789184\\
413	0.228728739311021\\
414	0.150475098876995\\
415	0.0649682915450312\\
416	-0.0228344202681307\\
417	-0.108213496784761\\
418	-0.186980298791002\\
419	-0.255709487667387\\
420	-0.311899257936155\\
421	-0.354631292289679\\
422	-0.381702989171209\\
423	-0.395306077313301\\
424	-0.396173724549872\\
425	-0.386299095405024\\
426	-0.368173251675693\\
427	-0.34458610757731\\
428	-0.318427047125714\\
429	-0.292496224888721\\
430	-0.269335063345135\\
431	-0.251081959976964\\
432	-0.239357027548323\\
433	-0.235754565378795\\
434	-0.239984869423071\\
435	-0.251094790930685\\
436	-0.268719919552755\\
437	-0.291321347480637\\
438	-0.316836383629516\\
439	-0.342783866545663\\
440	-0.366399733720519\\
441	-0.384800244994767\\
442	-0.395167882245551\\
443	-0.39495209993413\\
444	-0.382074111383267\\
445	-0.355698743425147\\
446	-0.314598439441782\\
447	-0.258522086226682\\
448	-0.189505765990705\\
449	-0.110065177609862\\
450	-0.0236903206986539\\
451	0.0653193346457865\\
452	0.15209794025541\\
453	0.231522470635965\\
454	0.298568736432045\\
455	0.348688862191046\\
456	0.378180644034282\\
457	0.383940824918907\\
458	0.366601029099969\\
459	0.324952931184668\\
460	0.261732099115609\\
461	0.180660727608609\\
462	0.0868161644544133\\
463	-0.0136928514003896\\
464	-0.114143941437717\\
465	-0.207681911237011\\
466	-0.287824983309252\\
467	-0.348955206519766\\
468	-0.386751877075418\\
469	-0.397954074540164\\
470	-0.382381741447338\\
471	-0.341766610618568\\
472	-0.278432879132142\\
473	-0.196880539653374\\
474	-0.102860498357295\\
475	-0.00293003421584096\\
476	0.0960503410237247\\
477	0.187435899387773\\
478	0.26528865756671\\
479	0.324799177756147\\
480	0.362604717230839\\
481	0.376402454220155\\
482	0.367874238402849\\
483	0.336856299879057\\
484	0.286856404380351\\
485	0.221854166656552\\
486	0.146485251851556\\
487	0.0656195144173915\\
488	-0.0160496786291388\\
489	-0.0943667945972286\\
490	-0.165994265521536\\
491	-0.228588807148775\\
492	-0.280862673083267\\
493	-0.321954242318096\\
494	-0.354149871652271\\
495	-0.376913071133476\\
496	-0.392340816631644\\
497	-0.401992795887963\\
498	-0.407195924701105\\
499	-0.408826925564258\\
500	-0.407174609207786\\
501	-0.401897423446451\\
502	-0.392080701069945\\
503	-0.376386490814681\\
504	-0.353278127685088\\
505	-0.320716376700747\\
506	-0.278254098807742\\
507	-0.226065653263893\\
508	-0.163924987854868\\
509	-0.0931512509485395\\
510	-0.0160722884644842\\
511	0.0640450551416587\\
512	0.143173442404936\\
513	0.216799849892682\\
514	0.280273702802765\\
515	0.329197066240531\\
516	0.359820360355234\\
517	0.369981952985655\\
518	0.356526381396038\\
519	0.321237758859204\\
520	0.265177518823361\\
521	0.191504842147649\\
522	0.104714108848068\\
523	0.0103319820209675\\
524	-0.0854779849559216\\
525	-0.176370013398569\\
526	-0.256291164948297\\
527	-0.319935899254304\\
528	-0.363140992838468\\
529	-0.383763652011806\\
530	-0.378988435810307\\
531	-0.351115459056537\\
532	-0.301738164653675\\
533	-0.234409245910098\\
534	-0.153748200231688\\
535	-0.0650491275579082\\
536	0.0261521401742642\\
537	0.114513137001744\\
538	0.195284447433566\\
539	0.264640595301035\\
540	0.319916002212515\\
541	0.360305926473581\\
542	0.383997356791717\\
543	0.3938019419395\\
544	0.391217063174851\\
545	0.379027796764355\\
546	0.360412279215549\\
547	0.338612019900377\\
548	0.316622726001593\\
549	0.296934397561092\\
550	0.28134338786496\\
551	0.270851022255004\\
552	0.265654072293157\\
553	0.26464628177271\\
554	0.267377289651008\\
555	0.273019648864074\\
556	0.279365961760072\\
557	0.284875716987357\\
558	0.288315419966878\\
559	0.288927170219532\\
560	0.286530527696965\\
561	0.281546630524985\\
562	0.274942186728079\\
563	0.268099822984853\\
564	0.262629422986123\\
565	0.260718248129753\\
566	0.262016295994218\\
567	0.269151146787426\\
568	0.281809225959248\\
569	0.299494730971503\\
570	0.320915088421435\\
571	0.34402998831339\\
572	0.366186497822532\\
573	0.384330440045449\\
574	0.39527618115543\\
575	0.396010558476837\\
576	0.384002544674157\\
577	0.358065359439732\\
578	0.316783172961777\\
579	0.259907093154354\\
580	0.189655348317815\\
581	0.108861049974585\\
582	0.0213968545131565\\
583	-0.0680728527724841\\
584	-0.154416794325121\\
585	-0.232412575752831\\
586	-0.297142466795162\\
587	-0.344376175548552\\
588	-0.370908475803261\\
589	-0.375399829392829\\
590	-0.356738375512028\\
591	-0.315338159560671\\
592	-0.254321424316951\\
593	-0.177542524320233\\
594	-0.0898966724079023\\
595	0.00302372521457584\\
596	0.0953205521624435\\
597	0.181197941190125\\
598	0.255354825574189\\
599	0.313332378199368\\
600	0.35179124475572\\
601	0.369276573807702\\
602	0.364500670594382\\
603	0.337569819317583\\
604	0.291113608183236\\
605	0.228282443393402\\
606	0.153061263170119\\
607	0.0699631411658621\\
608	-0.016295660247375\\
609	-0.101114482427472\\
610	-0.18029849316293\\
611	-0.250301837026939\\
612	-0.30841026002252\\
613	-0.352278833015746\\
614	-0.38285178418817\\
615	-0.398600567364697\\
616	-0.401176814252078\\
617	-0.392401831859217\\
618	-0.374672424845201\\
619	-0.350767432451277\\
620	-0.323644705747883\\
621	-0.296240491338099\\
622	-0.271281468849995\\
623	-0.25111775828239\\
624	-0.237583328824109\\
625	-0.231312027846763\\
626	-0.234541823624165\\
627	-0.245342722210853\\
628	-0.263359143403739\\
629	-0.286978508635416\\
630	-0.313992946432319\\
631	-0.341723549365943\\
632	-0.367180496677871\\
633	-0.387253512177984\\
634	-0.398924503201811\\
635	-0.399491489578551\\
636	-0.386790395076911\\
637	-0.359975863123229\\
638	-0.316815518472574\\
639	-0.259548437754471\\
640	-0.189204177939848\\
641	-0.108460281935446\\
642	-0.0209649201129939\\
643	0.0688463965243921\\
644	0.156009817570426\\
645	0.235353910745009\\
646	0.301861545992884\\
647	0.351046845579351\\
648	0.3793176130807\\
649	0.384869111470134\\
650	0.36505098290923\\
651	0.322249624215056\\
652	0.258179154821537\\
653	0.176661384173327\\
654	0.0828294306208485\\
655	-0.0172034134312762\\
656	-0.116763427718255\\
657	-0.209093463231856\\
658	-0.287849685808355\\
659	-0.34757760483838\\
660	-0.384127573940039\\
661	-0.394397691542645\\
662	-0.378335659999039\\
663	-0.337753127401595\\
664	-0.274994476780924\\
665	-0.194514637054782\\
666	-0.101956083994126\\
667	-0.00371383175091685\\
668	0.0935511628260016\\
669	0.183408972779011\\
670	0.260126393618999\\
671	0.319063076101296\\
672	0.356966684574855\\
673	0.372721898429298\\
674	0.365583621183465\\
675	0.336325403690762\\
676	0.288269564507903\\
677	0.225175850510744\\
678	0.151449740350054\\
679	0.0717477343672063\\
680	-0.00940612547846616\\
681	-0.0879576224891457\\
682	-0.160584888489999\\
683	-0.22486615275651\\
684	-0.279343273113968\\
685	-0.324059223739929\\
686	-0.357552481429788\\
687	-0.38239618261114\\
688	-0.399220076471502\\
689	-0.40933473560175\\
690	-0.41392122455218\\
691	-0.413842216259211\\
692	-0.409518746740294\\
693	-0.400885199970436\\
694	-0.387425797716505\\
695	-0.368286404813127\\
696	-0.342446858026258\\
697	-0.309508741304652\\
698	-0.267043425762461\\
699	-0.216539773963786\\
700	-0.157836221957191\\
701	-0.0921044511127087\\
702	-0.0213023316651031\\
703	0.051881813391754\\
704	0.124168264611421\\
705	0.191897535857796\\
706	0.251296109710654\\
707	0.298771979808356\\
708	0.331212333820451\\
709	0.345678561027623\\
710	0.341431960829208\\
711	0.318839894561869\\
712	0.27819522283757\\
713	0.221489822428552\\
714	0.151714989371327\\
715	0.0726579902358061\\
716	-0.011364900279117\\
717	-0.0958165646795151\\
718	-0.176262788578065\\
719	-0.248685000151495\\
720	-0.309756812495864\\
721	-0.357639286603563\\
722	-0.389249425623218\\
723	-0.406059030053871\\
724	-0.408331997214778\\
725	-0.397893443946248\\
726	-0.377374337820623\\
727	-0.34997677071062\\
728	-0.319203137325051\\
729	-0.288571130146056\\
730	-0.261336723572872\\
731	-0.240245834277448\\
732	-0.227332334824656\\
733	-0.223199338641286\\
734	-0.229821401487707\\
735	-0.244803851635002\\
736	-0.267186361682166\\
737	-0.294694840174571\\
738	-0.32449779044548\\
739	-0.3534240405264\\
740	-0.378202407791174\\
741	-0.395706405587313\\
742	-0.403187337407191\\
743	-0.398480409688074\\
744	-0.380170681415297\\
745	-0.348285129331278\\
746	-0.302544533272372\\
747	-0.243600968864432\\
748	-0.174214151180931\\
749	-0.0973236818959752\\
750	-0.0164542051911788\\
751	0.0644948147371228\\
752	0.141484623246688\\
753	0.210582971458841\\
754	0.268204542589129\\
755	0.311330738608844\\
756	0.337694866860454\\
757	0.346497501608518\\
758	0.335614000077609\\
759	0.307355734987716\\
760	0.26269428785395\\
761	0.204040043756549\\
762	0.134514361032737\\
763	0.0577541515463956\\
764	-0.0223150175170126\\
765	-0.101722439957331\\
766	-0.176696324790063\\
767	-0.243892170603698\\
768	-0.300590222164495\\
769	-0.345425380687951\\
770	-0.375610220629952\\
771	-0.392711600138587\\
772	-0.396889411063791\\
773	-0.389688647148349\\
774	-0.373324127648603\\
775	-0.350497225363399\\
776	-0.324181443445148\\
777	-0.297392491931217\\
778	-0.27295961881705\\
779	-0.253314813299637\\
780	-0.240315121387619\\
781	-0.2345342378221\\
782	-0.238061783872488\\
783	-0.24874843300833\\
784	-0.265986693931824\\
785	-0.287934400350508\\
786	-0.31223986564176\\
787	-0.336227917657501\\
788	-0.357108987918636\\
789	-0.372195226166883\\
790	-0.379107205879394\\
791	-0.375955652206138\\
792	-0.361484689989647\\
793	-0.335742766489433\\
794	-0.297245570604043\\
795	-0.248695716726496\\
796	-0.191172270413489\\
797	-0.126885659483453\\
798	-0.058441074570621\\
799	0.011347528788962\\
800	0.0796545616201181\\
801	0.14383557503549\\
802	0.201600770053949\\
803	0.251156130690067\\
804	0.291301918799667\\
805	0.322058499026067\\
806	0.341787930901548\\
807	0.352880199407067\\
808	0.355920917093907\\
809	0.352439117249082\\
810	0.344180333134423\\
811	0.332947930626348\\
812	0.320450406048451\\
813	0.308168105576774\\
814	0.297250795475113\\
815	0.288454440559951\\
816	0.282121700129213\\
817	0.278782928433895\\
818	0.276344432217284\\
819	0.275922877549997\\
820	0.276225366481448\\
821	0.276550794763608\\
822	0.276331157944448\\
823	0.275228249373117\\
824	0.273197585351032\\
825	0.270511303337528\\
826	0.267736073575786\\
827	0.265666895049954\\
828	0.265222543986202\\
829	0.267889473633884\\
830	0.273768563644658\\
831	0.28266980420445\\
832	0.295224428019367\\
833	0.310983220928249\\
834	0.32894375381382\\
835	0.347557050191996\\
836	0.364795110330823\\
837	0.378278285872162\\
838	0.385455376707007\\
839	0.383823532920328\\
840	0.371170167039343\\
841	0.34639218718567\\
842	0.306840518017065\\
843	0.254234406165248\\
844	0.189036398967946\\
845	0.113361651318964\\
846	0.0303452985955713\\
847	-0.056004818663553\\
848	-0.141040484546607\\
849	-0.219776461295549\\
850	-0.287243824699807\\
851	-0.338866764454467\\
852	-0.370832942055468\\
853	-0.379850012326182\\
854	-0.366288330152636\\
855	-0.328618037715718\\
856	-0.269211435373175\\
857	-0.191411630579355\\
858	-0.0999339955033471\\
859	-0.000580527815045656\\
860	0.100137932722814\\
861	0.195445375414831\\
862	0.278778010145432\\
863	0.344259765289914\\
864	0.387139611313775\\
865	0.404156631790322\\
866	0.393268825132379\\
867	0.356042760235497\\
868	0.294170369663019\\
869	0.211678375746776\\
870	0.114089963602051\\
871	0.00805045716060653\\
872	-0.09913341677377\\
873	-0.200007757189957\\
874	-0.287501948708649\\
875	-0.355436529315596\\
876	-0.398970994650868\\
877	-0.414958244081072\\
878	-0.40164420486023\\
879	-0.361013745263058\\
880	-0.29525415502659\\
881	-0.208907210762753\\
882	-0.107975000000963\\
883	0.00050360612778312\\
884	0.108951942049632\\
885	0.209793923123107\\
886	0.29599275359898\\
887	0.361550819724338\\
888	0.401934429141445\\
889	0.414392569191608\\
890	0.397611801076849\\
891	0.3540127892029\\
892	0.286164571724576\\
893	0.198892450017877\\
894	0.0983491458674922\\
895	-0.00843019125934301\\
896	-0.114032675606655\\
897	-0.21119483704931\\
898	-0.293313579995662\\
899	-0.354903802558775\\
900	-0.391969839177818\\
901	-0.402264664218578\\
902	-0.385954365823855\\
903	-0.34336149738541\\
904	-0.278311254286013\\
905	-0.195525451872747\\
906	-0.100834560527613\\
907	-0.000742958675334273\\
908	0.0980418854808198\\
909	0.189080234298563\\
910	0.266635808570897\\
911	0.326045058806306\\
912	0.363994739826608\\
913	0.378690509645676\\
914	0.369373505036019\\
915	0.338505150972331\\
916	0.288109544676667\\
917	0.221859614768424\\
918	0.144182924253172\\
919	0.0599089538560773\\
920	-0.0260898247941466\\
921	-0.109235315257665\\
922	-0.185535149321446\\
923	-0.251808733221964\\
924	-0.305835163805047\\
925	-0.346417353245093\\
926	-0.372829034569417\\
927	-0.386971784381397\\
928	-0.38973475053835\\
929	-0.383142731946565\\
930	-0.369564021543767\\
931	-0.351491852370748\\
932	-0.331340923669846\\
933	-0.311274084163968\\
934	-0.293070223234835\\
935	-0.27803977424684\\
936	-0.266989443831367\\
937	-0.260233301461612\\
938	-0.257108720126359\\
939	-0.258306729249934\\
940	-0.262484797277978\\
941	-0.268718681038394\\
942	-0.276034329109053\\
943	-0.28350465357765\\
944	-0.290327463132026\\
945	-0.295880163995445\\
946	-0.299749689796269\\
947	-0.301738765278535\\
948	-0.301851784604094\\
949	-0.30026511124005\\
950	-0.296752525397405\\
951	-0.29289044308815\\
952	-0.288526994622519\\
953	-0.284086969553977\\
954	-0.279966858074003\\
955	-0.276514897577353\\
956	-0.274018815665009\\
957	-0.272699513832965\\
958	-0.272708401762257\\
959	-0.27412606172849\\
960	-0.276960356338\\
961	-0.281142885445433\\
962	-0.285988830432868\\
963	-0.292440102134698\\
964	-0.299572652999188\\
965	-0.30693840413616\\
966	-0.313996012657151\\
967	-0.320126549914369\\
968	-0.324658536874285\\
969	-0.326902268489438\\
970	-0.326192119593645\\
971	-0.321934278165512\\
972	-0.313656267888115\\
973	-0.301053858263674\\
974	-0.284565507376054\\
975	-0.263150829281038\\
976	-0.237790239490212\\
977	-0.209135124126809\\
978	-0.178055843591138\\
979	-0.145594527807696\\
980	-0.112902317091745\\
981	-0.0811657558548254\\
982	-0.0515282431949542\\
983	-0.0250130909446489\\
984	-0.0024547560393487\\
985	0.0155558132208734\\
986	0.0292418590685972\\
987	0.0374054994280149\\
988	0.0409063084634927\\
989	0.040286751800124\\
990	0.0362957951095243\\
991	0.0298215244232248\\
992	0.0218156905852997\\
993	0.0132171086130164\\
994	0.0048807261307605\\
995	-0.00248160514995611\\
996	-0.00834396769683513\\
997	-0.0123898984565234\\
998	-0.0150505089304548\\
999	-0.0152392594454967\\
1000	-0.0138071666947559\\
1001	-0.0111546262002471\\
1002	-0.00775333374992111\\
1003	-0.00408998571003196\\
1004	-0.000614077592186487\\
1005	0.00230523896388959\\
1006	0.0044095582174614\\
1007	0.00556679048726989\\
1008	0.00577152732143234\\
1009	0.00513147930989287\\
1010	0.00437753756257977\\
1011	0.00258218038361363\\
1012	0.00061142424092942\\
1013	-0.00125346396317809\\
1014	-0.00277233971332671\\
1015	-0.00377148490393795\\
1016	-0.00415894599324267\\
1017	-0.00392887304088847\\
1018	-0.00315517078986897\\
1019	-0.00197599808746523\\
1020	-0.000571622852771823\\
1021	0.000861242086274626\\
1022	0.00160041810630895\\
1023	0.00267039823959252\\
1024	0.0033721138533471\\
1025	0.00363665644162094\\
1026	0.00345389583773787\\
1027	0.00287011796360939\\
1028	0.00197886908476308\\
1029	0.00090652721387639\\
1030	-0.000205326636790214\\
1031	-0.00121839199866728\\
1032	-0.00201466728114689\\
1033	-0.00250998824964511\\
1034	-0.00212774058580661\\
1035	-0.0020509410612322\\
1036	-0.0017317026145701\\
1037	-0.00122683877301724\\
1038	-0.000612583526587603\\
1039	2.60302131634506e-05\\
1040	0.000606278186009397\\
1041	0.00105794202388811\\
1042	0.00133138518307725\\
1043	0.0014026206628472\\
1044	0.00127490012186368\\
1045	0.000976807645371726\\
1046	2.2408935278716e-05\\
1047	-0.000346779259932254\\
1048	-0.00065791586799778\\
1049	-0.000871842885119677\\
1050	-0.000964172872144169\\
1051	-0.000927672977900042\\
1052	-0.000772312924795471\\
1053	-0.000523094494698773\\
1054	-0.000216056204341559\\
1055	0.000106940970781431\\
1056	0.000403916336707732\\
1057	0.000638204520537575\\
1058	0.000248072363220802\\
1059	0.000398975408808907\\
1060	0.000497197933638433\\
1061	0.000532888144974568\\
1062	0.000504364098228946\\
1063	0.000417928980701385\\
1064	0.000286712478470131\\
1065	0.00012872711098113\\
1066	-3.5584566884292e-05\\
1067	-0.000185986588255437\\
1068	-0.000304937223321329\\
1069	-0.000379628323527513\\
1070	-0.000938211550198804\\
1071	-0.000801087510968723\\
1072	-0.000568169236926067\\
1073	-0.000274114269349746\\
1074	4.00102373354814e-05\\
1075	0.000332251833619657\\
1076	0.000565200867393783\\
1077	0.000710632420907016\\
1078	0.000752797398044757\\
1079	0.000689990690795753\\
1080	0.000534257548782622\\
1081	0.00030934200393549\\
1082	-0.000487663689891892\\
1083	-0.000641484689394609\\
1084	-0.000710986583408642\\
1085	-0.000689952691577969\\
1086	-0.00058404841971295\\
1087	-0.000409679056841888\\
1088	-0.000191499263699676\\
1089	4.10576122895505e-05\\
1090	0.000257748625914949\\
1091	0.000431459154985877\\
1092	0.00054160203012496\\
1093	0.000576556801538217\\
};
\addplot[forget plot, color=white!15!black] table[row sep=crcr] {%
0	0\\
1200	0\\
};
\addlegendentry{data1}

\end{axis}
\end{tikzpicture}%
	}
	\caption{Signal non-modulé dans l'émetteur.}
\end{figure}
\begin{figure}[htbp]
	\centering
	\resizebox{\textwidth}{!}{%
		% !TEX root = ../report.tex
% This file was created by matlab2tikz.
%
%The latest updates can be retrieved from
%  http://www.mathworks.com/matlabcentral/fileexchange/22022-matlab2tikz-matlab2tikz
%where you can also make suggestions and rate matlab2tikz.
%
\definecolor{mycolor1}{rgb}{0.00000,0.44700,0.74100}%
%
\begin{tikzpicture}

\begin{axis}[%
width=15.5in,
height=8.535in,
at={(2.6in,1.152in)},
scale only axis,
xmin=0,
xmax=1200,
ymin=-250,
ymax=250,
axis background/.style={fill=white},
xmajorgrids,
ymajorgrids,
legend style={legend cell align=left, align=left, draw=white!15!black}
]
\addplot[ycomb, color=mycolor1, mark=o, mark options={solid, mycolor1}] table[row sep=crcr] {%
1	-1.00173390452018e-06\\
2	-1.59246992469754e-05\\
3	4.12705383211142e-05\\
4	-2.86106709401065e-06\\
5	-0.000250095635616808\\
6	0.000437976825549349\\
7	-7.79357431892358e-05\\
8	-0.00151639235444746\\
9	0.00155794680020716\\
10	-0.000589352757144085\\
11	-0.00406907991028622\\
12	0.00223927207909464\\
13	-0.00139860853486503\\
14	-0.00455431123327409\\
15	0.000848497378829476\\
16	-0.000809459281901629\\
17	0.000514442385387702\\
18	-0.00141955883558405\\
19	0.00163028588956293\\
20	0.00770817574838543\\
21	-0.00301223000997243\\
22	0.00325156650934916\\
23	0.010305875818644\\
24	-0.00319923844987289\\
25	0.00221036943638849\\
26	0.00447244488117375\\
27	-0.000322334355493679\\
28	-0.00105336799515453\\
29	-0.00746720020823068\\
30	0.00327565361683299\\
31	-0.00443162523651703\\
32	-0.0144369372219371\\
33	0.00345263258894086\\
34	-0.00404065238081624\\
35	-0.00842242083726528\\
36	0.00155251443501661\\
37	0.000482927205201193\\
38	0.00433218730239482\\
39	-0.000722079708592841\\
40	0.00428816462258168\\
41	0.0125876556031121\\
42	-0.00274361103792871\\
43	0.00401548258346211\\
44	0.0093256321237161\\
45	-0.00179164375512817\\
46	7.75672925363442e-05\\
47	-0.00339004564741549\\
48	0.000647447677252495\\
49	-0.00424279986064898\\
50	-0.0115421775642295\\
51	0.00066073724575223\\
52	-0.00385768996745648\\
53	-0.00525374957317264\\
54	-0.000151854741158169\\
55	0.00132497316524395\\
56	0.00672451387839438\\
57	-0.000107193915059232\\
58	0.00473079246616932\\
59	0.0103876366286069\\
60	-0.00026207001014411\\
61	0.00260893020558161\\
62	0.0021431871697406\\
63	0.000707712105065007\\
64	-0.00219048112957707\\
65	-0.00994772704637733\\
66	0.00166355232661933\\
67	-0.00482556743452449\\
68	-0.00992480289305599\\
69	-0.00135073834479015\\
70	-0.00121111909407582\\
71	0.00753000131743846\\
72	-0.00403152817678186\\
73	0.00617808945407362\\
74	0.0202738458217377\\
75	-0.00103556054954548\\
76	0.00646301086517047\\
77	0.00648556264052464\\
78	0.00305418749641599\\
79	-0.00351896474476306\\
80	-0.0233533459529556\\
81	0.0062041808309811\\
82	-0.0127747262615258\\
83	-0.0415015346512227\\
84	0.00702320583345876\\
85	-0.0116284185627746\\
86	-0.0221122957118071\\
87	-0.00143955082241521\\
88	0.0025800452424261\\
89	0.0321614994167049\\
90	-0.00974655186142072\\
91	0.0203063323157019\\
92	0.0637934677759172\\
93	-0.00544008088132718\\
94	0.0182624336762471\\
95	0.0286175286000155\\
96	-0.000168433472518344\\
97	-0.00588236816684711\\
98	-0.0322890644920348\\
99	-0.000211992448822782\\
100	-0.0215666753667928\\
101	-0.0468530095190711\\
102	-0.00340100791139231\\
103	-0.00808560872036922\\
104	0.024350068819497\\
105	-0.0193262800273586\\
106	0.0295977083569665\\
107	0.137500495441775\\
108	-0.0182701041853374\\
109	0.0566172453301299\\
110	0.10183699882202\\
111	0.0358667208466342\\
112	0.00153120743379384\\
113	-0.201089538410586\\
114	0.0873366507550149\\
115	-0.128773333649442\\
116	-0.548887192878005\\
117	0.0947251484367344\\
118	-0.203411508442483\\
119	-0.555794302018747\\
120	0.00128814126253114\\
121	-0.0999844675761882\\
122	0.254813976143934\\
123	-0.29641744426598\\
124	0.29838266261838\\
125	2.01551834308007\\
126	-0.631098777451302\\
127	0.89688287501271\\
128	3.88883370080713\\
129	-0.86461791665101\\
130	1.3161388475951\\
131	4.83998534507599\\
132	-0.882309499425108\\
133	1.28777395044367\\
134	3.89196116229804\\
135	-0.477145565646805\\
136	0.629956971148454\\
137	0.890495761589156\\
138	0.0841503528080733\\
139	-0.440118102613167\\
140	-2.66429879055259\\
141	0.587444123127029\\
142	-1.29887679634318\\
143	-5.05304541817419\\
144	0.842119586157166\\
145	-1.52556375197339\\
146	-4.78297674195145\\
147	0.554457182691016\\
148	-0.879292759535629\\
149	-1.62987228470139\\
150	0.0221905745744301\\
151	0.33901432877293\\
152	2.46840721581398\\
153	-0.503920197286947\\
154	1.34739404190003\\
155	5.32850468064891\\
156	-0.803905250207232\\
157	1.63499071824041\\
158	5.18448474537554\\
159	-0.549839107691854\\
160	0.953095320069934\\
161	1.82838455292205\\
162	-0.0410364538637961\\
163	-0.340886481025304\\
164	-2.55042998037802\\
165	0.467485133912905\\
166	-1.39779389017411\\
167	-5.56744950274338\\
168	0.762583553095049\\
169	-1.68397192080429\\
170	-5.38584952844952\\
171	0.520414928433436\\
172	-0.964884876011494\\
173	-1.851262003778\\
174	0.0374687578822054\\
175	0.368823362520332\\
176	2.69616748096107\\
177	-0.440103804701412\\
178	1.43842662707881\\
179	5.76362591963121\\
180	-0.714327950697638\\
181	1.70602885548238\\
182	5.49699974734666\\
183	-0.477921937463268\\
184	0.955306990338626\\
185	1.81142978859248\\
186	-0.0259681587403975\\
187	-0.399188317830057\\
188	-2.84295144035206\\
189	0.412614625970903\\
190	-1.46657243077236\\
191	-5.92290759854388\\
192	0.66471900829603\\
193	-1.71370285384295\\
194	-5.58308419369096\\
195	0.440768726709024\\
196	-0.942252712169276\\
197	-1.79093023541616\\
198	0.024070754702183\\
199	0.416311675653825\\
200	2.92520630091558\\
201	-0.375575989463089\\
202	1.46908263462463\\
203	5.99418775629206\\
204	-0.609966006524168\\
205	1.69847066434675\\
206	5.63400225201919\\
207	-0.413122285864271\\
208	0.940548570921798\\
209	1.89222938726319\\
210	-0.0423538353661046\\
211	-0.369551614670983\\
212	-2.74223239620824\\
213	0.321305397494181\\
214	-1.38962023700881\\
215	-5.82918902449825\\
216	0.559007156166003\\
217	-1.648829060748\\
218	-5.74998562633293\\
219	0.436176318201894\\
220	-1.02119455151728\\
221	-2.54024858707096\\
222	0.124602972584272\\
223	0.117941653902134\\
224	1.78678779469278\\
225	-0.253676580190114\\
226	1.12500529489603\\
227	5.3340148167475\\
228	-0.590026941741582\\
229	1.65980417064762\\
230	6.92516445676983\\
231	-0.728878250296395\\
232	1.65844263300064\\
233	6.62031540692275\\
234	-0.729891484030923\\
235	1.3562581878661\\
236	5.52905828958799\\
237	-0.663770170220704\\
238	1.0803488939969\\
239	4.67325117035596\\
240	-0.57646171453259\\
241	0.989709632850244\\
242	4.47853249352968\\
243	-0.534907505711574\\
244	1.0604754725177\\
245	4.84795377078696\\
246	-0.535418890866636\\
247	1.18684369047409\\
248	5.27947616200357\\
249	-0.543132219575824\\
250	1.24666180940504\\
251	5.45427041828679\\
252	-0.552076927219643\\
253	1.22579043091653\\
254	5.37260445224056\\
255	-0.547301722106952\\
256	1.17289294878802\\
257	5.22787491555273\\
258	-0.531548303973172\\
259	1.14601373603295\\
260	5.1962174377325\\
261	-0.521256392532865\\
262	1.16308189744861\\
263	5.34329035990692\\
264	-0.515192751321522\\
265	1.21011346721107\\
266	5.53237185844171\\
267	-0.509053570481192\\
268	1.24152991675626\\
269	5.65898804322948\\
270	-0.504479533376127\\
271	1.24386365306737\\
272	5.65476010473329\\
273	-0.494071394426904\\
274	1.21558000891223\\
275	5.57148668913803\\
276	-0.484677956828243\\
277	1.18645105732832\\
278	5.52295801396854\\
279	-0.479587090632237\\
280	1.18491458900039\\
281	5.6118593762637\\
282	-0.471617677164561\\
283	1.22082864590442\\
284	5.79361933555277\\
285	-0.464265479269892\\
286	1.26405174254227\\
287	5.98827102077057\\
288	-0.454561831758385\\
289	1.28414287401315\\
290	5.99731601015471\\
291	-0.433908716279324\\
292	1.23923738967275\\
293	5.77954096540561\\
294	-0.423810959458779\\
295	1.15982958763595\\
296	5.54803311135676\\
297	-0.423853838226662\\
298	1.1243205307022\\
299	5.59650893714629\\
300	-0.429092951397526\\
301	1.1897541000212\\
302	6.10688203337654\\
303	-0.454489682519468\\
304	1.36050705372385\\
305	6.98335320135722\\
306	-0.450357044087641\\
307	1.5439306098586\\
308	7.48611764396432\\
309	-0.398821815906723\\
310	1.53117286598502\\
311	6.94011408208246\\
312	-0.31486650247241\\
313	1.21820076939396\\
314	4.86792213716249\\
315	-0.157842288468114\\
316	0.568084933356172\\
317	1.37642891384351\\
318	-0.00159468539261824\\
319	-0.266865020973882\\
320	-2.57556387177859\\
321	0.166090971867644\\
322	-1.0367121206484\\
323	-6.11628876823675\\
324	0.315698121571035\\
325	-1.58396140594945\\
326	-8.4280856975204\\
327	0.407478901064105\\
328	-1.82117819547586\\
329	-9.31143963624973\\
330	0.43426125882548\\
331	-1.79513323877552\\
332	-8.87103155218553\\
333	0.382442655099251\\
334	-1.54764834808286\\
335	-7.31544980651909\\
336	0.2738715029278\\
337	-1.11204150735083\\
338	-4.64839057592108\\
339	0.113097297392099\\
340	-0.457965352922975\\
341	-0.9591443310559\\
342	-0.0249484869833184\\
343	0.335247669425654\\
344	3.1300033448786\\
345	-0.160655664810986\\
346	1.09813059169627\\
347	6.87580214384518\\
348	-0.248174392450964\\
349	1.64771496971733\\
350	9.07588978203521\\
351	-0.277164846062668\\
352	1.78500434557405\\
353	9.32301110291778\\
354	-0.312585209346373\\
355	1.57540654841055\\
356	8.20051405306076\\
357	-0.319700142763658\\
358	1.24913320591852\\
359	6.79874217598295\\
360	-0.318790476843124\\
361	1.06166311730665\\
362	6.53611656352664\\
363	-0.338824578870497\\
364	1.21663477556582\\
365	7.87612798298434\\
366	-0.292886234671931\\
367	1.59441191168636\\
368	9.48704503899275\\
369	-0.221537944236948\\
370	1.79385194474712\\
371	9.73308257528927\\
372	-0.14779871815078\\
373	1.52302653512693\\
374	6.85638493728039\\
375	-0.0103545328475278\\
376	0.592652446344316\\
377	0.743997217096851\\
378	0.0345377121093601\\
379	-0.69634214479061\\
380	-5.89989848269543\\
381	0.0429578163546303\\
382	-1.67463823380841\\
383	-10.162836193474\\
384	0.0446165574187411\\
385	-1.86453468062979\\
386	-9.19830965899533\\
387	-0.0307810893241233\\
388	-1.00341483520558\\
389	-2.80488731764061\\
390	-0.0142825291944773\\
391	0.483574437221504\\
392	5.3180052748514\\
393	0.0177943702037391\\
394	1.70774083964672\\
395	10.9940648982094\\
396	0.0190871907368927\\
397	2.01659773967224\\
398	10.4781693512041\\
399	0.054659116180717\\
400	1.12588730871247\\
401	3.5568833849126\\
402	0.00249779130689446\\
403	-0.451031306866805\\
404	-5.51989984590129\\
405	-0.0520174201747546\\
406	-1.74825856420817\\
407	-11.7648333753041\\
408	-0.0634251686836807\\
409	-2.05107144820792\\
410	-11.247565268182\\
411	-0.0570868215901997\\
412	-1.15342801628873\\
413	-4.08513361657123\\
414	0.00715219618647857\\
415	0.39595348719916\\
416	5.43490918582046\\
417	0.0713814925895997\\
418	1.67310393561851\\
419	11.9713176887431\\
420	0.0743487146191701\\
421	1.99692240717455\\
422	12.0191970908279\\
423	0.0397235700071101\\
424	1.25557638176682\\
425	5.60771588601758\\
426	-0.00850588301731198\\
427	-0.103136761560543\\
428	-3.73428309761577\\
429	-0.0328336622332717\\
430	-1.37485861599361\\
431	-11.6860550061954\\
432	-0.00953512539726355\\
433	-2.06192313921909\\
434	-15.4226105388269\\
435	0.0503668761493239\\
436	-2.06380650198444\\
437	-14.9004620693342\\
438	0.122033098235891\\
439	-1.65628368505577\\
440	-12.2106993780831\\
441	0.169499505224814\\
442	-1.2483145596091\\
443	-10.1092744023715\\
444	0.178688924902042\\
445	-1.1437321046906\\
446	-10.4357654648647\\
447	0.15587374939464\\
448	-1.4042749613958\\
449	-13.0276282507606\\
450	0.0807953128064204\\
451	-1.80030527379008\\
452	-15.611682310855\\
453	0.0104410156452811\\
454	-1.95530859185315\\
455	-15.7132439664027\\
456	-0.0443841412168515\\
457	-1.60329545496891\\
458	-10.736184112963\\
459	-0.0897769706841559\\
460	-0.583900925339975\\
461	-0.895822725411456\\
462	0.00258207911646978\\
463	0.760319539365324\\
464	9.80555451456617\\
465	0.104417284417762\\
466	1.79072601455366\\
467	17.0073563459157\\
468	0.165373224628993\\
469	2.00116898372266\\
470	15.4599289528425\\
471	0.186819152520441\\
472	1.06577453614741\\
473	4.68547492309561\\
474	0.00954843013136615\\
475	-0.52522996061794\\
476	-9.17375119951237\\
477	-0.164840890228197\\
478	-1.82596938309384\\
479	-19.3480564159387\\
480	-0.226752335483249\\
481	-2.16177953565234\\
482	-18.7181348100682\\
483	-0.211456974920711\\
484	-1.20702104096689\\
485	-6.55340049826294\\
486	0.000450833740475387\\
487	0.450588053804674\\
488	9.39946902610059\\
489	0.19448428812843\\
490	1.77222457332003\\
491	21.0550875037478\\
492	0.235890659247406\\
493	2.12192114242024\\
494	21.4131373536215\\
495	0.184246793851823\\
496	1.31805998480947\\
497	10.0509352272268\\
498	0.00760339774829824\\
499	-0.0989567950813442\\
500	-6.15090176961175\\
501	-0.126681898503781\\
502	-1.3569120141298\\
503	-20.6891050085875\\
504	-0.136496397028598\\
505	-2.07728475246195\\
506	-29.4286881790443\\
507	-0.0858482110448104\\
508	-2.2698482292303\\
509	-32.3544408476934\\
510	-0.0514901042463575\\
511	-2.16543798180967\\
512	-31.3807558016657\\
513	-0.056636941647082\\
514	-1.90699173122907\\
515	-27.4463367857311\\
516	-0.102467761224776\\
517	-1.4266179230252\\
518	-17.5972636886055\\
519	-0.128348377492094\\
520	-0.483566702172195\\
521	-0.532567885287126\\
522	0.00769541570115955\\
523	0.752944877772754\\
524	19.2000591749755\\
525	0.169486368508401\\
526	1.76419004673339\\
527	34.4523873210333\\
528	0.276152218861643\\
529	2.03157388757021\\
530	33.5067147199619\\
531	0.27459331532521\\
532	1.17114826033302\\
533	12.5874369908106\\
534	0.0335685925371717\\
535	-0.393127775898728\\
536	-18.6390455890949\\
537	-0.220165097042564\\
538	-1.77187449866151\\
539	-46.0309681832156\\
540	-0.345729595997183\\
541	-2.24616777400939\\
542	-50.6570254559514\\
543	-0.309280578120584\\
544	-1.45000912146957\\
545	-25.6927052043543\\
546	-0.0498827532753941\\
547	0.108525042862091\\
548	17.8416808225987\\
549	0.189791048618205\\
550	1.56679332986144\\
551	63.9145035782871\\
552	0.292742140482217\\
553	2.35194704775541\\
554	93.2728387720438\\
555	0.266440925299149\\
556	2.30524243823294\\
557	100.660136003121\\
558	0.128663199745104\\
559	1.83844855036518\\
560	98.2387003550547\\
561	0.0298934565847469\\
562	1.47567281821671\\
563	102.569953012019\\
564	0.0435144314878683\\
565	1.41468305168984\\
566	131.191812042314\\
567	0.101147923945559\\
568	1.56680564535201\\
569	201.27007967937\\
570	0.166160153158907\\
571	1.67447403767227\\
572	355.878152738814\\
573	0.154884917393499\\
574	1.60260618883269\\
575	1395.02896231797\\
576	0.0879520876285652\\
577	1.55224879975095\\
578	-708.448066671496\\
579	0.0863741534108368\\
580	1.67842564724597\\
581	-313.150469611034\\
582	0.146165409040498\\
583	1.99903276595056\\
584	-222.054142168531\\
585	0.274209898977529\\
586	2.20056033335457\\
587	-155.74021526227\\
588	0.337195060736395\\
589	1.80403979282531\\
590	-82.3043022816147\\
591	0.220196273431766\\
592	0.676949183752424\\
593	-6.90355352043025\\
594	-0.0411339119981072\\
595	-0.847714234688739\\
596	50.6934956925751\\
597	-0.326845026090073\\
598	-2.03761758248672\\
599	72.3204337884438\\
600	-0.465278922223071\\
601	-2.19225995319738\\
602	55.7301691379157\\
603	-0.337555563093364\\
604	-1.16604238158127\\
605	16.3423590265569\\
606	-0.0128115332776319\\
607	0.49525139455642\\
608	-23.6596454676842\\
609	0.319790316467862\\
610	1.90084835192313\\
611	-45.2127233321086\\
612	0.483431826161137\\
613	2.29274780714593\\
614	-41.0147570887893\\
615	0.390571695447758\\
616	1.46912969772601\\
617	-17.9025877811316\\
618	0.0798524953370462\\
619	-0.0761570114810788\\
620	9.83971400287588\\
621	-0.230012263827991\\
622	-1.57437899862276\\
623	31.068916025926\\
624	-0.418247320539773\\
625	-2.47155849159084\\
626	38.5697181175346\\
627	-0.454072947459969\\
628	-2.50764526380033\\
629	34.7357422299217\\
630	-0.301192847150395\\
631	-2.03554831019424\\
632	27.13430105028\\
633	-0.170561905844212\\
634	-1.5551279360077\\
635	20.3496774405421\\
636	-0.114549453241076\\
637	-1.36377212543406\\
638	19.2581195262308\\
639	-0.132035022652082\\
640	-1.67603186835435\\
641	22.5177974292692\\
642	-0.300071914556708\\
643	-2.16672707321593\\
644	24.9610567817264\\
645	-0.421445633314032\\
646	-2.36005719416122\\
647	23.6339886233759\\
648	-0.428304552095434\\
649	-1.93529412534833\\
650	14.9800181920562\\
651	-0.285817001017394\\
652	-0.687798315952664\\
653	1.02837680419158\\
654	0.0733569228322224\\
655	0.927954443007044\\
656	-12.1519929028136\\
657	0.415181896430908\\
658	2.15802474186262\\
659	-19.3438397542634\\
660	0.574278847669763\\
661	2.351069658297\\
662	-16.2155101375584\\
663	0.440731280341947\\
664	1.23126397210095\\
665	-4.58684449997149\\
666	0.0136098153452331\\
667	-0.63034380300748\\
668	8.78899127999389\\
669	-0.441443850772076\\
670	-2.17651505843166\\
671	16.8005081376888\\
672	-0.648968546415017\\
673	-2.51818972795961\\
674	15.0791310466619\\
675	-0.4937924723267\\
676	-1.4007183389071\\
677	5.08250783660139\\
678	-0.026281042704809\\
679	0.507533637991738\\
680	-6.68470855522623\\
681	0.438066914200332\\
682	2.07200673042417\\
683	-14.1051172378583\\
684	0.633408327240566\\
685	2.48719225285032\\
686	-13.5042294431312\\
687	0.49394067145553\\
688	1.56979486362688\\
689	-6.23196611798182\\
690	0.0948562747176778\\
691	-0.0579755207057678\\
692	3.16456229748347\\
693	-0.282950790777987\\
694	-1.58080880506199\\
695	10.9140348797281\\
696	-0.502675791204991\\
697	-2.51257318355228\\
698	14.7728758709122\\
699	-0.554639853606\\
700	-2.77625128050242\\
701	15.2913536916238\\
702	-0.505778648043902\\
703	-2.63887340006232\\
704	13.8112191270096\\
705	-0.459621440371808\\
706	-2.28238856221213\\
707	10.8307521476114\\
708	-0.405769378724985\\
709	-1.62419532954112\\
710	6.25355099950618\\
711	-0.239428366251225\\
712	-0.552265533615646\\
713	0.370258163107358\\
714	0.0560845939027609\\
715	0.791268579004662\\
716	-5.62925082979736\\
717	0.389362289427385\\
718	1.94538159621609\\
719	-9.38692734585232\\
720	0.595986955230687\\
721	2.29846199920461\\
722	-8.88330202578202\\
723	0.517858006260883\\
724	1.52945019498021\\
725	-4.21982421132436\\
726	0.145404494011608\\
727	-0.0339531022342506\\
728	2.4469526125962\\
729	-0.292424041118238\\
730	-1.68694933976778\\
731	8.48074501048358\\
732	-0.621500751492635\\
733	-2.75572857170766\\
734	11.1721396601302\\
735	-0.711966356828148\\
736	-2.83020788906474\\
737	10.5615511661882\\
738	-0.519021714186662\\
739	-2.27398550865889\\
740	8.48882588038677\\
741	-0.31883228583914\\
742	-1.67853568436664\\
743	6.45777633877875\\
744	-0.220490957911603\\
745	-1.47115342970322\\
746	6.43767745970561\\
747	-0.27439253967115\\
748	-1.89603083622954\\
749	7.90084110129077\\
750	-0.513607756453891\\
751	-2.48330940283924\\
752	8.93050397497666\\
753	-0.656228140622087\\
754	-2.64595187093961\\
755	8.47532057417531\\
756	-0.601231406278597\\
757	-2.07422591086808\\
758	5.30746113387139\\
759	-0.33917573341371\\
760	-0.681046277643018\\
761	0.31989170209798\\
762	0.122542450935292\\
763	0.977522187060864\\
764	-4.31065370390885\\
765	0.520694418466068\\
766	2.17286714562518\\
767	-6.93608449705124\\
768	0.677704307132746\\
769	2.38790390779889\\
770	-6.30205849779761\\
771	0.527484896705288\\
772	1.48157058517021\\
773	-2.73643697154738\\
774	0.12967291940472\\
775	-0.152144925047929\\
776	2.07431515984207\\
777	-0.351299997779349\\
778	-1.7784714476206\\
779	5.98925184363665\\
780	-0.674572611943148\\
781	-2.74864187337662\\
782	7.98677714526436\\
783	-0.742257545694002\\
784	-2.91186989700795\\
785	7.74884711984395\\
786	-0.634950336952303\\
787	-2.41187775422406\\
788	6.21840971044518\\
789	-0.401495630312121\\
790	-1.81762806510618\\
791	5.16700459112273\\
792	-0.288109693688197\\
793	-1.6673611702884\\
794	4.92163658121627\\
795	-0.378480394601966\\
796	-1.94193178110113\\
797	5.62527448806158\\
798	-0.529459034455115\\
799	-2.45165134359527\\
800	6.5057130102768\\
801	-0.694391411413631\\
802	-2.65521963843503\\
803	6.01872246751407\\
804	-0.657003011267126\\
805	-2.10253292682025\\
806	4.11941857149557\\
807	-0.366819817196467\\
808	-0.919045249482457\\
809	0.976174761403044\\
810	0.0221256909119336\\
811	0.616658908235326\\
812	-2.49300108420263\\
813	0.436053120487084\\
814	1.93815429944964\\
815	-4.89502644253969\\
816	0.675916203046919\\
817	2.61894572119961\\
818	-6.01268241758421\\
819	0.700415431709957\\
820	2.70378435729427\\
821	-5.91963009879666\\
822	0.61558309411366\\
823	2.41124508822259\\
824	-5.20272518423593\\
825	0.500147984085687\\
826	2.09742039705985\\
827	-4.64274112346579\\
828	0.452025281391404\\
829	1.99097741873241\\
830	-4.3553905450496\\
831	0.483609661182812\\
832	2.00330526445215\\
833	-4.25576905134597\\
834	0.499907919054652\\
835	2.04531615973364\\
836	-4.28457062305189\\
837	0.509963893899242\\
838	2.05758583291632\\
839	-4.21461997437488\\
840	0.492133768973922\\
841	2.0777985233169\\
842	-4.45699429507827\\
843	0.507679192257727\\
844	2.32632685178263\\
845	-4.88567662555505\\
846	0.643554573110271\\
847	2.64232596255453\\
848	-5.10588521488585\\
849	0.738609549245454\\
850	2.73881802452769\\
851	-4.80810625036755\\
852	0.720575635955526\\
853	2.2442449808364\\
854	-3.07967933219631\\
855	0.453645609222626\\
856	0.833344686889158\\
857	-0.282121879972106\\
858	-0.109518829766623\\
859	-1.00749367776453\\
860	2.55761130390983\\
861	-0.636996738139699\\
862	-2.49823748370263\\
863	4.40617046294767\\
864	-0.91373240903277\\
865	-2.84289621598439\\
866	3.83035799557626\\
867	-0.725258917911776\\
868	-1.56512961576447\\
869	1.22148864823774\\
870	-0.0489820204605476\\
871	0.631345693287274\\
872	-1.97256158631708\\
873	0.653027561263256\\
874	2.56560023086624\\
875	-4.25707223530796\\
876	1.02641175886176\\
877	3.13193579455433\\
878	-3.96758773001043\\
879	0.822096920162261\\
880	1.80055459982298\\
881	-1.39896304441312\\
882	0.0756956634656066\\
883	-0.608989936490612\\
884	1.87614128840595\\
885	-0.705031559184193\\
886	-2.70276704473357\\
887	4.12133927862605\\
888	-1.09057075408932\\
889	-3.24853233691937\\
890	3.84120130435103\\
891	-0.838434987497543\\
892	-1.82859483633757\\
893	1.30775327596888\\
894	-0.0610226761810577\\
895	0.694566584635212\\
896	-1.8875781740068\\
897	0.755913094867472\\
898	2.80761933910932\\
899	-3.91973666044\\
900	1.12008456559793\\
901	3.26908994381815\\
902	-3.60262381718983\\
903	0.827598321531231\\
904	1.78218039279616\\
905	-1.1721272346101\\
906	0.0345105828235148\\
907	-0.756116351965793\\
908	1.8165084436245\\
909	-0.773461842384525\\
910	-2.82040485294058\\
911	3.65584928559544\\
912	-1.11451326865966\\
913	-3.22007316114338\\
914	3.34221050124706\\
915	-0.803971649431153\\
916	-1.76143984863129\\
917	1.14981894598355\\
918	-0.0443350728929544\\
919	0.670785183559602\\
920	-1.57812715965456\\
921	0.723212950831402\\
922	2.65987891807489\\
923	-3.29039347871431\\
924	1.05985083964334\\
925	3.13791770077118\\
926	-3.18663604246244\\
927	0.816905339245716\\
928	1.95166716824369\\
929	-1.44793430431118\\
930	0.168969564480872\\
931	-0.180139920021344\\
932	0.939770283750104\\
933	-0.523591347987854\\
934	-2.18498293283085\\
935	2.88816945645941\\
936	-0.955948242518024\\
937	-3.285895681145\\
938	3.78753545428649\\
939	-1.02431731713602\\
940	-3.37080247696989\\
941	3.71978034241825\\
942	-0.856943417421261\\
943	-2.86215474766443\\
944	3.15935747453716\\
945	-0.650121737705763\\
946	-2.32399433098685\\
947	2.6583107802951\\
948	-0.550354276328237\\
949	-2.10603854101122\\
950	2.4672938349571\\
951	-0.578141426796698\\
952	-2.21719208697386\\
953	2.54569612993899\\
954	-0.668739993452629\\
955	-2.44627959800679\\
956	2.69245408940044\\
957	-0.732217770030703\\
958	-2.59066416308348\\
959	2.75391057844291\\
960	-0.740659700603123\\
961	-2.5725431027628\\
962	2.67131656586213\\
963	-0.700648847060309\\
964	-2.45522834999903\\
965	2.5484150402258\\
966	-0.658712023381177\\
967	-2.3643362855957\\
968	2.46007693715972\\
969	-0.65028188204409\\
970	-2.37261750410496\\
971	2.45636711293329\\
972	-0.679763182921399\\
973	-2.48309348967167\\
974	2.54932553597626\\
975	-0.730662775558385\\
976	-2.66453504670439\\
977	2.66267288655548\\
978	-0.795467124133625\\
979	-2.79930106289577\\
980	2.66392771857526\\
981	-0.817074008715281\\
982	-2.76441551209397\\
983	2.49906160125957\\
984	-0.76819540066767\\
985	-2.45285323170314\\
986	2.04366800218159\\
987	-0.612697125489119\\
988	-1.80396275227309\\
989	1.39194029361384\\
990	-0.360436006859793\\
991	-0.977565116946664\\
992	0.679880741294631\\
993	-0.102538071825243\\
994	-0.194106056878754\\
995	0.0743060433269816\\
996	0.090552783154642\\
997	0.309271991432165\\
998	-0.237094379483216\\
999	0.157288812819813\\
1000	0.417164796021816\\
1001	-0.275902731460072\\
1002	0.104411579826545\\
1003	0.244529604418771\\
1004	-0.147205467215003\\
1005	0.0110147901407804\\
1006	-0.00766397725908666\\
1007	0.0197036422048419\\
1008	-0.0547142823469377\\
1009	-0.155986504601433\\
1010	0.0953617015753484\\
1011	-0.0580550047281016\\
1012	-0.131858128160946\\
1013	0.0719311452221353\\
1014	-0.0158689338584015\\
1015	-0.0227173079299983\\
1016	0.01237153812806\\
1017	0.0200880014652017\\
1018	0.0550016197765387\\
1019	-0.0307683983861062\\
1020	0.0245854123508784\\
1021	0.0553545228974241\\
1022	-0.0262098345935612\\
1023	0.00700748930200515\\
1024	0.00255799290964893\\
1025	0.0052163654801142\\
1026	-0.0131285318440816\\
1027	-0.0398269867982907\\
1028	0.018380148945741\\
1029	-0.0169496949780792\\
1030	-0.0311842125100343\\
1031	0.00743281441183183\\
1032	-0.00249049975085436\\
1033	0.00816102581577735\\
1034	-0.00836447469443048\\
1035	0.0122278386626685\\
1036	0.0343601838404498\\
1037	-0.0157529783570296\\
1038	0.013094656494596\\
1039	0.0262275095406828\\
1040	-0.00865155787065452\\
1041	0.00232435677268404\\
1042	-0.00344928849531603\\
1043	0.0040532041963803\\
1044	-0.00836553611662787\\
1045	-0.022724589022809\\
1046	0.00875506071433012\\
1047	-0.00852145850743778\\
1048	-0.0159269124591595\\
1049	0.00590445493754004\\
1050	-0.000284492181570948\\
1051	0.00257637473889278\\
1052	-0.000293895692101825\\
1053	0.00542344574212564\\
1054	0.0134432044556603\\
1055	-0.0063101420385227\\
1056	0.00436083027575962\\
1057	0.0100208874536764\\
1058	-0.00478173322310035\\
1059	-7.24055027696228e-06\\
1060	-0.00253255433425552\\
1061	0.00187918973089829\\
1062	-0.00382054585793522\\
1063	-0.0102226967336096\\
1064	0.00296304426642077\\
1065	-0.00348176839974667\\
1066	-0.00418928672386132\\
1067	-0.000876541385041501\\
1068	0.00126106878510259\\
1069	0.00702156613375327\\
1070	-0.00291263363405447\\
1071	0.004511784687063\\
1072	0.0101939213583435\\
1073	-0.00322960686899768\\
1074	0.0023167887947687\\
1075	0.00388643073131506\\
1076	-0.00244783803187205\\
1077	-0.00161246653932253\\
1078	-0.00354108338307413\\
1079	0.000186333607451595\\
1080	-0.00235769101672559\\
1081	-0.00451212725849096\\
1082	0.00168925915046202\\
1083	-0.000291723418176746\\
1084	0.000109475043180527\\
1085	-0.000182673056826558\\
1086	0.00112143493676206\\
1087	0.00384634563197487\\
1088	-0.00278799926996282\\
1089	0.00098848759572933\\
1090	0.00419336837462523\\
1091	-0.00255778477113401\\
1092	0.000988601536986708\\
1093	0.00248548269547368\\
1094	0.000195350849792554\\
1095	0.000557041256568909\\
1096	-0.00110030999219895\\
1097	0.00114013314636478\\
1098	-0.00155415036023557\\
1099	-0.00380437583541375\\
1100	-0.00108792030471541\\
1101	-0.00213825017421046\\
1102	-0.00075035308432463\\
1103	-0.00194298277363371\\
1104	0.001206092495362\\
1105	0.00535707986954129\\
1106	-0.000388047855316246\\
1107	0.00355405091912099\\
1108	0.00614080115258126\\
1109	-0.000793155396164392\\
1110	0.00077152200670808\\
1111	0.000533910543224769\\
1112	-0.00222654562233496\\
1113	-0.00250419879304623\\
1114	-0.00372249817157119\\
1115	-0.000285765089809885\\
1116	-0.00118493880499366\\
1117	-0.00176807488074127\\
1118	0.00227672980910425\\
1119	0.00160592444302642\\
1120	0.00204932786533158\\
1121	0.000686298389539515\\
1122	0.00107032872958429\\
1123	0.00178144400892702\\
1124	-0.00187379510022096\\
1125	-0.00104637775893448\\
1126	-0.00115232851769831\\
1127	-0.000698233848112782\\
1128	-0.000854302626634139\\
1129	-0.00152406977426952\\
1130	0.0014724592878257\\
1131	0.000705717120184488\\
1132	0.000675846018537655\\
1133	0.00058890881822645\\
1134	0.000653995715044089\\
1135	0.00123748835139861\\
1136	-0.00114216535070618\\
1137	-0.000491483979819079\\
1138	-0.000413879211188987\\
1139	-0.000463460437177165\\
1140	-0.000493177860402983\\
1141	-0.000982581623702914\\
1142	0.000880995480152584\\
1143	0.000350982959953203\\
1144	0.000263423543222468\\
1145	0.000354439064234664\\
1146	0.000370208882386246\\
1147	0.000771233287758541\\
1148	-0.000676855660196321\\
1149	-0.000255261567687115\\
1150	-0.000173198410307457\\
1151	-0.000267968181074824\\
};
\addplot[forget plot, color=white!15!black] table[row sep=crcr] {%
0	0\\
1200	0\\
};
\addlegendentry{data1}

\end{axis}
\end{tikzpicture}%
	}
	\caption{Signal démodulé dans le receveur sans avoir pris en compte
			le déphasage de la porteuse.}
\end{figure}
\begin{figure}[htbp]
	\centering
	\resizebox{\textwidth}{!}{%
		% !TEX root = ../report.tex
% This file was created by matlab2tikz.
%
%The latest updates can be retrieved from
%  http://www.mathworks.com/matlabcentral/fileexchange/22022-matlab2tikz-matlab2tikz
%where you can also make suggestions and rate matlab2tikz.
%
\definecolor{mycolor1}{rgb}{0.00000,0.44700,0.74100}%
%
\begin{tikzpicture}

\begin{axis}[%
width=10.242in,
height=8.535in,
at={(5.229in,1.152in)},
scale only axis,
xmin=0,
xmax=1200,
ymin=-3,
ymax=3,
axis background/.style={fill=white},
xmajorgrids,
ymajorgrids,
legend style={legend cell align=left, align=left, draw=white!15!black}
]
\addplot[ycomb, color=mycolor1, mark=o, mark options={solid, mycolor1}] table[row sep=crcr] {%
1	-0.00860007078229813\\
2	-0.00833267736314476\\
3	-0.00800447538100436\\
4	-0.0101997881811323\\
5	-0.00746831634356537\\
6	-0.00695036169760577\\
7	-0.00790072167514887\\
8	-0.00628493913160162\\
9	-0.00549991723250866\\
10	-0.00602867707040973\\
11	-0.00468248322704904\\
12	-0.00350985490415099\\
13	-0.00396680149510304\\
14	-0.00253709951824285\\
15	-0.000976474830737893\\
16	-0.00124996879729465\\
17	0.00021592669311212\\
18	0.00159643008030813\\
19	0.00231061467816353\\
20	0.00335617927561906\\
21	0.00342820534653294\\
22	0.00599779402123505\\
23	0.00522639466701231\\
24	0.00360986821623394\\
25	0.00481293697796777\\
26	0.00232478389383781\\
27	0.000385970769980677\\
28	-0.00190048603834297\\
29	-0.00354159447923653\\
30	-0.00414713647728751\\
31	-0.00706472142647861\\
32	-0.00657420444824792\\
33	-0.00417776621768444\\
34	-0.00702642641830077\\
35	-0.00398603622648328\\
36	-0.00181775172806174\\
37	0.00091987552237807\\
38	0.00210994096776567\\
39	0.000896656058842902\\
40	0.00763565864555204\\
41	0.00595116278018973\\
42	0.00357674182968271\\
43	0.00667031505793557\\
44	0.00427035182360877\\
45	0.00225972959812654\\
46	0.000133292170079366\\
47	-0.0015658051860743\\
48	-0.000801688748913758\\
49	-0.00761725738836053\\
50	-0.00542131975636895\\
51	-0.000856648342167805\\
52	-0.00671133804586408\\
53	-0.00243557413976044\\
54	0.000203594540673621\\
55	0.00221935126101327\\
56	0.00304956893685611\\
57	0.000140948371763765\\
58	0.00803760468064926\\
59	0.00470915658101103\\
60	0.000342370148981222\\
61	0.00452471422617182\\
62	0.000979529876400867\\
63	-0.000957047164038516\\
64	-0.00373268551838269\\
65	-0.0045102476251841\\
66	-0.00230417857791361\\
67	-0.00804487847724342\\
68	-0.00443133236368234\\
69	0.00185445407470386\\
70	-0.00203028247153586\\
71	0.00335063422277915\\
72	0.00553886140199706\\
73	0.0104511578351967\\
74	0.00904515512463218\\
75	0.00146029471035966\\
76	0.0108113161645545\\
77	0.00287520817248225\\
78	-0.00438741348952187\\
79	-0.00580656420001922\\
80	-0.0102369359499492\\
81	-0.00889971981370716\\
82	-0.0211049254074011\\
83	-0.0181088293571337\\
84	-0.0101240645163342\\
85	-0.0192721756661376\\
86	-0.00964154934184086\\
87	0.00211783630799701\\
88	0.00424229423445782\\
89	0.0139419773366045\\
90	0.0145626893369189\\
91	0.0330798507393863\\
92	0.0274090225825849\\
93	0.0081561273064984\\
94	0.0297190297106258\\
95	0.0122326740121847\\
96	0.000254404545031303\\
97	-0.00957302979321507\\
98	-0.0137639184530741\\
99	0.000325618480568157\\
100	-0.0348838068671402\\
101	-0.0198589218812756\\
102	0.00529637182169799\\
103	-0.0129883430160389\\
104	0.010243023281195\\
105	0.030293744313131\\
106	0.0474365348408742\\
107	0.0575271684531704\\
108	0.0288961102957919\\
109	0.0905924349197381\\
110	0.0424349747603946\\
111	-0.057563646479451\\
112	0.00243759474359861\\
113	-0.0833155668733971\\
114	-0.141979798625406\\
115	-0.203864391224663\\
116	-0.22586096516921\\
117	-0.15530444797288\\
118	-0.321084248169084\\
119	-0.227415380591958\\
120	-0.00213311018030391\\
121	-0.157421971478119\\
122	0.103760213748025\\
123	0.497480745967244\\
124	0.467677687358378\\
125	0.815960270091939\\
126	1.07234190956918\\
127	1.39906014169265\\
128	1.56413786116465\\
129	1.48354436047901\\
130	2.0463467077886\\
131	1.93538646573768\\
132	1.53012942392219\\
133	1.99613818923539\\
134	1.54796612155971\\
135	0.838098598438416\\
136	0.972422825515742\\
137	0.352079260857445\\
138	-0.149615059458117\\
139	-0.676460545160021\\
140	-1.04669921600556\\
141	-1.05559361127252\\
142	-1.98948997777092\\
143	-1.97325785766248\\
144	-1.53022679997283\\
145	-2.32896893351883\\
146	-1.85708064993181\\
147	-1.02013354373506\\
148	-1.33707290775525\\
149	-0.628974971690026\\
150	-0.0413258039355539\\
151	0.51344251459232\\
152	0.94651315494727\\
153	0.949052829876932\\
154	2.03345414296398\\
155	2.03062339066613\\
156	1.53166202089476\\
157	2.45899622493831\\
158	1.96381131687796\\
159	1.06064144660735\\
160	1.42800282892332\\
161	0.688230141846576\\
162	0.0801319574528471\\
163	-0.508783658144737\\
164	-0.953836096871649\\
165	-0.923604107770527\\
166	-2.07884009000616\\
167	-2.06894985925233\\
168	-1.52472274669425\\
169	-2.49569217954475\\
170	-1.98886216437862\\
171	-1.05356009174337\\
172	-1.42470423164094\\
173	-0.679217522084109\\
174	-0.076799133617968\\
175	0.542563890555627\\
176	0.982704933443274\\
177	0.913062111712846\\
178	2.10851180198192\\
179	2.08698766088902\\
180	1.50029280205047\\
181	2.49199019534507\\
182	1.97743123187897\\
183	1.01651944811199\\
184	1.39036152785682\\
185	0.647291894391339\\
186	0.0559343266146143\\
187	-0.578873050662952\\
188	-1.00903999260481\\
189	-0.899919215814453\\
190	-2.11920514107986\\
191	-2.08800089778146\\
192	-1.46817743549052\\
193	-2.46762974775001\\
194	-1.95487126921716\\
195	-0.986136865613924\\
196	-1.35195389029497\\
197	-0.622775869661977\\
198	-0.0545528736890519\\
199	0.59519910727464\\
200	1.01014240011452\\
201	0.862202855458721\\
202	2.09298210186348\\
203	2.05549255617909\\
204	1.41857940344755\\
205	2.41135455438082\\
206	1.918428155137\\
207	0.973516727076734\\
208	1.33062404830041\\
209	0.639750657489629\\
210	0.10113507342311\\
211	-0.520980461275501\\
212	-0.920484636078741\\
213	-0.777461373043389\\
214	-1.95223174882276\\
215	-1.94256502849601\\
216	-1.37081931079091\\
217	-2.30837367985302\\
218	-1.9022485537723\\
219	-1.08416346424319\\
220	-1.4247169335538\\
221	-0.834213134283884\\
222	-0.313955008493818\\
223	0.163975530667696\\
224	0.582427434600925\\
225	0.647963344501269\\
226	1.55872405015719\\
227	1.72570321557745\\
228	1.52800218450035\\
229	2.29181498755581\\
230	2.22361058259012\\
231	1.91404219851195\\
232	2.28207276516484\\
233	2.10955711318684\\
234	1.94376268590871\\
235	1.85986065023879\\
236	1.7482959398528\\
237	1.79278615556096\\
238	1.47645048288408\\
239	1.46624056436431\\
240	1.57928639150393\\
241	1.34798148802905\\
242	1.39416541889533\\
243	1.48665506269071\\
244	1.43945635200535\\
245	1.49725437974808\\
246	1.50979290039146\\
247	1.60552127543026\\
248	1.61752964379704\\
249	1.55406456122712\\
250	1.68074465926871\\
251	1.65763957291218\\
252	1.60310207555082\\
253	1.64703870752275\\
254	1.61956049536173\\
255	1.61305867779711\\
256	1.57065964031302\\
257	1.56300551571147\\
258	1.59032556353386\\
259	1.52951124910899\\
260	1.54066870273792\\
261	1.58333015240576\\
262	1.54709502328994\\
263	1.5710201411165\\
264	1.58902496276557\\
265	1.60428016492152\\
266	1.61286835500393\\
267	1.59453341334368\\
268	1.64044235182398\\
269	1.63569573804572\\
270	1.60505040301643\\
271	1.6380578226741\\
272	1.62037309792919\\
273	1.5968926247714\\
274	1.59549931555143\\
275	1.58259580402272\\
276	1.59166410284855\\
277	1.55211120021975\\
278	1.55499431271794\\
279	1.60048870935441\\
280	1.54497871412781\\
281	1.56595949941454\\
282	1.59969271878976\\
283	1.58655651834157\\
284	1.60213117675941\\
285	1.60084667736101\\
286	1.63732366581587\\
287	1.6408966035258\\
288	1.5936504062693\\
289	1.65788716791886\\
290	1.62826338582296\\
291	1.54702906748415\\
292	1.59466947806511\\
293	1.55454734480552\\
294	1.53693860889493\\
295	1.48760538380413\\
296	1.47824519456217\\
297	1.56376279516071\\
298	1.43735555008209\\
299	1.47698014594145\\
300	1.61089329178922\\
301	1.51605404983404\\
302	1.5961698937038\\
303	1.73657586521926\\
304	1.72800217824695\\
305	1.80749240755889\\
306	1.75176888580979\\
307	1.95460945925024\\
308	1.91854342383895\\
309	1.57960094574234\\
310	1.9321820589524\\
311	1.76089198034851\\
312	1.27012418140029\\
313	1.53227606324757\\
314	1.22266707690853\\
315	0.648635642653074\\
316	0.712243165164584\\
317	0.342186468763757\\
318	0.00667758943375523\\
319	-0.333507907797555\\
320	-0.633680684169859\\
321	-0.708877328612979\\
322	-1.29144165304917\\
323	-1.48908199558091\\
324	-1.37370993844964\\
325	-1.96682585371985\\
326	-2.03017758390394\\
327	-1.80820740248697\\
328	-2.25414009948281\\
329	-2.21889522856918\\
330	-1.96580517058866\\
331	-2.21480127514017\\
332	-2.0909703527708\\
333	-1.76657754638228\\
334	-1.90336671738078\\
335	-1.70532031134306\\
336	-1.29129717380849\\
337	-1.36328202534816\\
338	-1.07150567216572\\
339	-0.544484991122802\\
340	-0.559647029716263\\
341	-0.218593261128027\\
342	0.122681446625105\\
343	0.408381961468576\\
344	0.705165079170531\\
345	0.807205483991667\\
346	1.33344870423341\\
347	1.53106280798343\\
348	1.27455087322804\\
349	1.99447333443217\\
350	1.99715159681493\\
351	1.45551240686043\\
352	2.15383024376895\\
353	2.02701394692917\\
354	1.67917505770818\\
355	1.894929649197\\
356	1.76134649944681\\
357	1.7575177051097\\
358	1.49775108620743\\
359	1.44230738138779\\
360	1.79423391157037\\
361	1.26896666783452\\
362	1.36928760530615\\
363	1.95326609410533\\
364	1.44963413433586\\
365	1.62911236994576\\
366	1.7302267524835\\
367	1.89380598833146\\
368	1.93708174226138\\
369	1.34178829503482\\
370	2.12402871530464\\
371	1.96136713082615\\
372	0.918254108305993\\
373	1.79772025232159\\
374	1.36334356930972\\
375	0.0660259303775883\\
376	0.697360838368918\\
377	0.145945404560225\\
378	-0.226160376734718\\
379	-0.816817324138153\\
380	-1.1415031118604\\
381	-0.289043874084228\\
382	-1.95825830308463\\
383	-1.93893103078864\\
384	-0.308666521870569\\
385	-2.17354206266307\\
386	-1.73008956353521\\
387	0.219097426042917\\
388	-1.16608059580803\\
389	-0.519976862577971\\
390	0.104669925360303\\
391	0.560226546062007\\
392	0.971441892192958\\
393	-0.134364238518715\\
394	1.97231562406743\\
395	1.97839765008502\\
396	-0.148616402410671\\
397	2.32182562586873\\
398	1.85699975517369\\
399	-0.439208355271369\\
400	1.29229855772682\\
401	0.620650441151537\\
402	-0.0207314182190494\\
403	-0.516100103064255\\
404	-0.9480587682998\\
405	0.44636717641231\\
406	-1.99431578332815\\
407	-1.98832601352163\\
408	0.563258559393188\\
409	-2.33255400021964\\
410	-1.86992633958501\\
411	0.52522571462368\\
412	-1.30769267050084\\
413	-0.667879071193577\\
414	-0.0682506434817123\\
415	0.447533173698472\\
416	0.873505453632174\\
417	-0.707357424459124\\
418	1.88526069407528\\
419	1.89080898784703\\
420	-0.766091941588691\\
421	2.24325501503506\\
422	1.86491165595861\\
423	-0.42620837918851\\
424	1.40614809764852\\
425	0.854445595937654\\
426	0.0951745329489928\\
427	-0.115152497238494\\
428	-0.55853855620386\\
429	0.383764697106703\\
430	-1.53035187890199\\
431	-1.71508733522646\\
432	0.116626093769367\\
433	-2.28812920046279\\
434	-2.22006678075172\\
435	-0.645934263397768\\
436	-2.28324884322664\\
437	-2.1028448209043\\
438	-1.64447200893477\\
439	-1.82682334854927\\
440	-1.6886841646609\\
441	-2.4057346991436\\
442	-1.37266562667447\\
443	-1.36936847604702\\
444	-2.67818365563451\\
445	-1.2538489833209\\
446	-1.38388344520325\\
447	-2.4742189272218\\
448	-1.53480973871298\\
449	-1.69039239413157\\
450	-1.3626393884052\\
451	-1.96169402679686\\
452	-1.98097914640172\\
453	-0.187780407290703\\
454	-2.12414601645474\\
455	-1.94873719699488\\
456	0.854756205756714\\
457	-1.73647188350049\\
458	-1.30056192646385\\
459	1.86010119399588\\
460	-0.630492112885905\\
461	-0.105930134651208\\
462	-0.0578713586379013\\
463	0.818510177411162\\
464	1.13108133867904\\
465	-2.54769807168779\\
466	1.92196565510349\\
467	1.91238154686415\\
468	-4.42569774331666\\
469	2.14136092215178\\
470	1.69330706046815\\
471	-5.5333595992066\\
472	1.13700437143377\\
473	0.499491253922156\\
474	-0.316465895710399\\
475	-0.558647679012281\\
476	-0.951047620599572\\
477	6.198040655569\\
478	-1.93630921173789\\
479	-1.94888509974529\\
480	9.8446970172428\\
481	-2.28552674696561\\
482	-1.83017675438087\\
483	10.8529480602404\\
484	-1.27228492209615\\
485	-0.621354985076097\\
486	-0.0282686705985859\\
487	0.473527019436597\\
488	0.863277431102518\\
489	-15.6515754810347\\
490	1.85686482822059\\
491	1.87101267498726\\
492	-26.4547651272012\\
493	2.21660386466299\\
494	1.83880185555734\\
495	-33.9892502528444\\
496	1.37275226902991\\
497	0.832948307191024\\
498	-3.92396025358671\\
499	-0.10275467191918\\
500	-0.491229309494273\\
501	-82.662293097901\\
502	-1.40477700487253\\
503	-1.58981857940301\\
504	-27.3539502975786\\
505	-2.14413556283261\\
506	-2.17223492931023\\
507	-10.1777585976245\\
508	-2.33590009066229\\
509	-2.28984002208073\\
510	-4.33883255632944\\
511	-2.22180089241093\\
512	-2.12521154372894\\
513	-3.70483954477962\\
514	-1.95079133961493\\
515	-1.77475801874579\\
516	-5.48101269006117\\
517	-1.45503339085873\\
518	-1.08384701393246\\
519	-5.81007634408407\\
520	-0.491728719012061\\
521	-0.0311606315990421\\
522	0.302086287158408\\
523	0.763373131904382\\
524	1.06402557336173\\
525	5.87558181637584\\
526	1.78329885241877\\
527	1.80236593188648\\
528	8.57462951369017\\
529	2.04746745746561\\
530	1.64855152592431\\
531	7.72330218939716\\
532	1.17679931769687\\
533	0.579972991556378\\
534	0.863158349041825\\
535	-0.393850045063587\\
536	-0.800347741614487\\
537	-5.21526906092562\\
538	-1.76985316464908\\
539	-1.83165581913159\\
540	-7.59358615118302\\
541	-2.23693839092341\\
542	-1.85571615106163\\
543	-6.33362071873098\\
544	-1.43976142669436\\
545	-0.859744361645573\\
546	-0.957023930858068\\
547	0.107438052014929\\
548	0.54025704690487\\
549	3.42564939053522\\
550	1.54649527627923\\
551	1.73124645440347\\
552	4.98945863777964\\
553	2.31458687792232\\
554	2.22748187056414\\
555	4.30226112689467\\
556	2.26189233515146\\
557	2.08002367740412\\
558	1.97402847960627\\
559	1.79852494557262\\
560	1.71271328247211\\
561	0.436943914293708\\
562	1.43934527471778\\
563	1.45570897760461\\
564	0.607394288513277\\
565	1.37576487555236\\
566	1.43499651154001\\
567	1.35121559934984\\
568	1.51918464192007\\
569	1.54403998401159\\
570	2.12855482670156\\
571	1.61876722069635\\
572	1.56308968940002\\
573	1.90609804725969\\
574	1.54469771910633\\
575	1.53479807336152\\
576	1.0415613126631\\
577	1.49172531467816\\
578	1.56190870292093\\
579	0.985810413495482\\
580	1.60820192082214\\
581	1.72939618689678\\
582	1.61006568996601\\
583	1.90971871073156\\
584	1.96597637683408\\
585	2.91907664546676\\
586	2.09601238682456\\
587	1.89970041584753\\
588	3.47329015945279\\
589	1.7132383153781\\
590	1.28029107594372\\
591	2.19717956165065\\
592	0.640971672760612\\
593	0.1306623657486\\
594	-0.398031239402652\\
595	-0.80028250060736\\
596	-1.13106140522476\\
597	-3.07013211346654\\
598	-1.91790702525693\\
599	-1.85939718497106\\
600	-4.24655160309974\\
601	-2.05734852558718\\
602	-1.62304559373915\\
603	-2.9961367025114\\
604	-1.09104082224078\\
605	-0.531946328547932\\
606	-0.110681248108375\\
607	0.462022459646619\\
608	0.851541313484158\\
609	2.69114882529393\\
610	1.76805353916887\\
611	1.78349982030433\\
612	3.9658043150177\\
613	2.12625137318556\\
614	1.76022981554405\\
615	3.1255566734342\\
616	1.35840299194918\\
617	0.830713036792333\\
618	0.623787543259844\\
619	-0.0702082781468215\\
620	-0.491017203894705\\
621	-1.75508077326361\\
622	-1.4470961684562\\
623	-1.65958772247718\\
624	-3.11917598231779\\
625	-2.26500095275347\\
626	-2.19640256652776\\
627	-3.31163880018094\\
628	-2.29125103455798\\
629	-2.10122080121431\\
630	-2.14936844447428\\
631	-1.85437142330425\\
632	-1.73801788239329\\
633	-1.19158335367299\\
634	-1.41250439419037\\
635	-1.37622691398078\\
636	-0.783842196308263\\
637	-1.23501853222386\\
638	-1.37158783303952\\
639	-0.885371397603518\\
640	-1.51328768874322\\
641	-1.68499849602238\\
642	-1.97269961309386\\
643	-1.95052065327343\\
644	-1.95829899533775\\
645	-2.71748143268854\\
646	-2.11824171292617\\
647	-1.94023374920904\\
648	-2.70987593258611\\
649	-1.73183340599423\\
650	-1.2845737120651\\
651	-1.7751347970476\\
652	-0.61365724145219\\
653	-0.0919642432973301\\
654	0.447402958143351\\
655	0.825460848854291\\
656	1.13156448475733\\
657	2.48755082326324\\
658	1.91395064230674\\
659	1.87298027585576\\
660	3.38132825821491\\
661	2.07894832800151\\
662	1.63048547271091\\
663	2.55104813570112\\
664	1.08550698272938\\
665	0.478379088824329\\
666	0.0774677700444056\\
667	-0.554066035605474\\
668	-0.949686370718379\\
669	-2.47176334043834\\
670	-1.9074256632533\\
671	-1.87883853915046\\
672	-3.57561913986119\\
673	-2.2002680037567\\
674	-1.74357591650388\\
675	-2.67791729341755\\
676	-1.22022072846997\\
677	-0.607068785713378\\
678	-0.140328156480698\\
679	0.440810666064686\\
680	0.824060861617828\\
681	2.30362840144925\\
682	1.79422666867017\\
683	1.79313907769385\\
684	3.28126773092305\\
685	2.14730413735999\\
686	1.76901319605514\\
687	2.52133742890776\\
688	1.35121455726847\\
689	0.840605590415118\\
690	0.477231653697069\\
691	-0.0497533579343915\\
692	-0.439223602255241\\
693	-1.40341565893083\\
694	-1.35254743773589\\
695	-1.55767467591075\\
696	-2.45854140022354\\
697	-2.14331678545158\\
698	-2.16673075041405\\
699	-2.67555286403649\\
700	-2.36113021381151\\
701	-2.30343960276732\\
702	-2.4069751151259\\
703	-2.23754744373957\\
704	-2.13554596019403\\
705	-2.15830533508319\\
706	-1.92945560455473\\
707	-1.71809896403098\\
708	-1.8805438300706\\
709	-1.36890694854761\\
710	-1.01720008111995\\
711	-1.09536491047921\\
712	-0.46405878324611\\
713	-0.0617251374662207\\
714	0.253332407850266\\
715	0.66288357519721\\
716	0.961352866821907\\
717	1.73679075962875\\
718	1.62482011959958\\
719	1.64148310913115\\
720	2.62577178135152\\
721	1.91392007425017\\
722	1.58994601266106\\
723	2.25390426526495\\
724	1.26971602786449\\
725	0.772715031908361\\
726	0.625292105625467\\
727	-0.0281018101842847\\
728	-0.458244247340281\\
729	-1.24271711353001\\
730	-1.39199987974786\\
731	-1.62363445270999\\
732	-2.61051815864334\\
733	-2.26701628149873\\
734	-2.18582933517798\\
735	-2.95624386582815\\
736	-2.32121982745492\\
737	-2.11097092314127\\
738	-2.13073813863687\\
739	-1.85936230812861\\
740	-1.73273299875722\\
741	-1.29430579693863\\
742	-1.36830822114966\\
743	-1.34572657305226\\
744	-0.885237966794682\\
745	-1.19560288971775\\
746	-1.36917321525382\\
747	-1.08968043694036\\
748	-1.53620291592914\\
749	-1.71446390159246\\
750	-2.01779679176283\\
751	-2.0058883909568\\
752	-1.9766601062853\\
753	-2.55081862557458\\
754	-2.13073346306341\\
755	-1.91290235590249\\
756	-2.31261682317543\\
757	-1.66522520732946\\
758	-1.22120335936463\\
759	-1.29116306583265\\
760	-0.545081860918868\\
761	-0.0750164540694315\\
762	0.46173503620398\\
763	0.779970781811071\\
764	1.03000849611236\\
765	1.94219784531379\\
766	1.72842236103928\\
767	1.6883074760302\\
768	2.50269618827795\\
769	1.89363773142593\\
770	1.56227673191056\\
771	1.92879953107175\\
772	1.17128902529646\\
773	0.690719232472736\\
774	0.469554683360576\\
775	-0.119911039109549\\
776	-0.533011752359251\\
777	-1.2598660187703\\
778	-1.39735522803108\\
779	-1.56635062043832\\
780	-2.39625282736249\\
781	-2.15295298779697\\
782	-2.12546358712627\\
783	-2.61194236889862\\
784	-2.27375158343075\\
785	-2.09796870178717\\
786	-2.21360454785878\\
787	-1.87749662541088\\
788	-1.7125286225961\\
789	-1.38687616107479\\
790	-1.41052195954858\\
791	-1.44714913554706\\
792	-0.9861786644814\\
793	-1.2898926698714\\
794	-1.40159234498341\\
795	-1.28388266991726\\
796	-1.49763157504094\\
797	-1.62861256075005\\
798	-1.78008588737083\\
799	-1.88484172908769\\
800	-1.91450834782619\\
801	-2.31409464444633\\
802	-2.03497835715862\\
803	-1.80004587918852\\
804	-2.1704660354602\\
805	-1.60636147340697\\
806	-1.25188015712244\\
807	-1.20139859170349\\
808	-0.699965174122139\\
809	-0.301394054196104\\
810	0.0718486955274616\\
811	0.468189330371762\\
812	0.781887875943406\\
813	1.40406559361478\\
814	1.46689720645847\\
815	1.55929562506171\\
816	2.15826255038965\\
817	1.97592516001396\\
818	1.94505277833917\\
819	2.21802922795712\\
820	2.03351088920778\\
821	1.94440816405072\\
822	1.93345639591687\\
823	1.80777354332179\\
824	1.73498291798721\\
825	1.55818010151164\\
826	1.56752365243892\\
827	1.57164176452509\\
828	1.39697263774578\\
829	1.48326444322562\\
830	1.49646257174654\\
831	1.48272336190451\\
832	1.48771819672761\\
833	1.48396228702608\\
834	1.5206471876742\\
835	1.51409422377684\\
836	1.5160238293017\\
837	1.53916020670309\\
838	1.5183325659349\\
839	1.51307451484515\\
840	1.47389168767049\\
841	1.52836266023764\\
842	1.62329621203183\\
843	1.50883319194161\\
844	1.70571028479823\\
845	1.80503724284624\\
846	1.89818083486734\\
847	1.93121198506345\\
848	1.91333703225691\\
849	2.16220949028171\\
850	1.99532291378949\\
851	1.82729193676578\\
852	2.09374038721417\\
853	1.62976178133292\\
854	1.18688462570119\\
855	1.30842729426463\\
856	0.603225581900394\\
857	0.110246735353876\\
858	-0.313573968021738\\
859	-0.72693561106532\\
860	-1.01332193749189\\
861	-1.81065173801517\\
862	-1.79673138005659\\
863	-1.76977150858718\\
864	-2.57864136864915\\
865	-2.03799680809744\\
866	-1.55954678705559\\
867	-2.03219976510233\\
868	-1.1183637187209\\
869	-0.504095461156248\\
870	-0.13628188972699\\
871	0.449663269331314\\
872	0.825050272712709\\
873	1.80421115315498\\
874	1.82135135890641\\
875	1.80447349700537\\
876	2.81616235125918\\
877	2.21615044911381\\
878	1.70419695763616\\
879	2.24008596532033\\
880	1.26990540552443\\
881	0.608861182056567\\
882	0.204853413855194\\
883	-0.428105574298731\\
884	-0.827298665051936\\
885	-1.8951161548355\\
886	-1.89374733440889\\
887	-1.84113955500148\\
888	-2.9117917541045\\
889	-2.26866404360532\\
890	-1.73833967330354\\
891	-2.22371315352733\\
892	-1.27281983890605\\
893	-0.599488885462778\\
894	-0.160778086984837\\
895	0.481866375574595\\
896	0.876430202452406\\
897	1.9785938575556\\
898	1.9413822121322\\
899	1.84330188235281\\
900	2.91277778617241\\
901	2.25297579453189\\
902	1.71576013067098\\
903	2.13830722493241\\
904	1.22415062979005\\
905	0.565304108748518\\
906	0.0885969167487616\\
907	-0.517633095919245\\
908	-0.887129297418462\\
909	-1.97307013303704\\
910	-1.92438376915098\\
911	-1.80781093451148\\
912	-2.82518797369783\\
913	-2.18972396936137\\
914	-1.67335298344406\\
915	-2.02526633760956\\
916	-1.19379946870219\\
917	-0.582836528604813\\
918	-0.110991224455179\\
919	0.45308891887514\\
920	0.809837740379444\\
921	1.79939868430644\\
922	1.79058400486458\\
923	1.70930043988117\\
924	2.62087199622103\\
925	2.10524976801899\\
926	1.6756914882691\\
927	2.00785441228795\\
928	1.30494866215638\\
929	0.770688006421395\\
930	0.412807573871936\\
931	-0.120038213892882\\
932	-0.506287640217344\\
933	-1.27153785141431\\
934	-1.45102784740146\\
935	-1.57478945895865\\
936	-2.30774359725349\\
937	-2.17468225640057\\
938	-2.09006444640568\\
939	-2.45822957168574\\
940	-2.22323739281884\\
941	-2.0773177029232\\
942	-2.04452745981561\\
943	-1.88127482323584\\
944	-1.78544771619635\\
945	-1.5420775847266\\
946	-1.52228809082174\\
947	-1.52018873305294\\
948	-1.29790293164322\\
949	-1.37475958844405\\
950	-1.42770084023771\\
951	-1.35562517465391\\
952	-1.44230919965991\\
953	-1.49049054590116\\
954	-1.5591416813176\\
955	-1.58581185532767\\
956	-1.59499585933103\\
957	-1.69749355342998\\
958	-1.67356678672067\\
959	-1.65056437625822\\
960	-1.707429413128\\
961	-1.65606261458474\\
962	-1.61980477058373\\
963	-1.60619023192477\\
964	-1.57501208681095\\
965	-1.56331309719634\\
966	-1.50169177413312\\
967	-1.5113840965237\\
968	-1.52667744962894\\
969	-1.47431877089573\\
970	-1.51134136394888\\
971	-1.54205357328592\\
972	-1.53273675366266\\
973	-1.57613252589408\\
974	-1.61891610698956\\
975	-1.63856075053534\\
976	-1.68531640059398\\
977	-1.71039126786626\\
978	-1.77426570316009\\
979	-1.76427159514776\\
980	-1.73087183887582\\
981	-1.81269061463555\\
982	-1.73608263596924\\
983	-1.64236972644524\\
984	-1.69517548772945\\
985	-1.53491819132923\\
986	-1.35844774311973\\
987	-1.34488166407989\\
988	-1.12482027767269\\
989	-0.935792143853466\\
990	-0.787002413985408\\
991	-0.607349125760227\\
992	-0.46228067668939\\
993	-0.222718814208119\\
994	-0.120161165049196\\
995	-0.051097517723488\\
996	0.195664227197271\\
997	0.190762536082872\\
998	0.164887781061399\\
999	0.338110881830087\\
1000	0.256379128418135\\
1001	0.194045721493583\\
1002	0.223293095426454\\
1003	0.149735226019194\\
1004	0.104699015157305\\
1005	0.0234358987174406\\
1006	-0.00467583533950067\\
1007	-0.0141718288407808\\
1008	-0.115823881562718\\
1009	-0.0948198736252632\\
1010	-0.069359293765852\\
1011	-0.122276270205446\\
1012	-0.079858548644248\\
1013	-0.0529041684282175\\
1014	-0.0332558683358992\\
1015	-0.0137078166067206\\
1016	-0.00920089823401464\\
1017	0.0418878471447387\\
1018	0.0330657594711227\\
1019	0.0231385848921685\\
1020	0.0510119760233929\\
1021	0.0331544906112718\\
1022	0.019930294639263\\
1023	0.0144681447406611\\
1024	0.00152640354514496\\
1025	-0.00401076794911459\\
1026	-0.0269733297856119\\
1027	-0.0236767861420711\\
1028	-0.0142893203604689\\
1029	-0.0346545588503495\\
1030	-0.0184692719629749\\
1031	-0.0058426810984765\\
1032	-0.00506730457618857\\
1033	0.00481530492722678\\
1034	0.00664794805981505\\
1035	0.0247596215545708\\
1036	0.0201972784141404\\
1037	0.0126588939097304\\
1038	0.0263878395599316\\
1039	0.0153584352771198\\
1040	0.00702920162415698\\
1041	0.00466164732271429\\
1042	-0.00201217674472619\\
1043	-0.0033295351420758\\
1044	-0.0166981605395067\\
1045	-0.0132060641038499\\
1046	-0.00727131444162942\\
1047	-0.0169292689111495\\
1048	-0.0092202715462722\\
1049	-0.00495790040378693\\
1050	-0.000562542147168717\\
1051	0.00148576399507857\\
1052	0.000249500473630467\\
1053	0.0106741031920158\\
1054	0.00772264602190721\\
1055	0.00541593907004567\\
1056	0.00854294118018223\\
1057	0.00573437129872443\\
1058	0.00414928374741712\\
1059	-1.41189556307342e-05\\
1060	-0.00144360505312089\\
1061	-0.00164857465964846\\
1062	-0.00741582862866374\\
1063	-0.00580441845100735\\
1064	-0.0026279901204356\\
1065	-0.00672740828795237\\
1066	-0.00236935685839323\\
1067	0.000785965305628584\\
1068	0.00242555184495424\\
1069	0.00395562393122675\\
1070	0.00264034266618675\\
1071	0.00863881445041446\\
1072	0.00572013497262742\\
1073	0.00295982516084286\\
1074	0.00441607263847146\\
1075	0.00217216703587142\\
1076	0.00226798343814666\\
1077	-0.00305980906596832\\
1078	-0.00197128284980045\\
1079	-0.000174537355845159\\
1080	-0.00445403989488653\\
1081	-0.00250182980797018\\
1082	-0.001599679699729\\
1083	-0.000548670655381998\\
1084	6.04572271608288e-05\\
1085	0.000174884255860118\\
1086	0.00209989344562314\\
1087	0.00211558773729001\\
1088	0.00269841414012905\\
1089	0.0018428353382387\\
1090	0.00229714417463505\\
1091	0.00250276631372003\\
1092	0.00183500817159002\\
1093	0.00135603614889254\\
1094	-0.00019322254285735\\
1095	0.00103055012005342\\
1096	-0.000599097770874798\\
1097	-0.00114570187904212\\
1098	-0.00289927824712073\\
1099	-0.00211788480823067\\
1100	0.00116368550338181\\
1101	-0.00433312673461951\\
1102	-0.000480145126416779\\
1103	0.00256109699339313\\
1104	0.00359480164889612\\
1105	0.0055629025057364\\
1106	0.000808994518215289\\
1107	0.0763041751147935\\
1108	0.038560743646403\\
1109	0.00398145202793962\\
1110	-0.00437267066426272\\
1111	-0.00162180193475947\\
1112	-0.0634466012280758\\
1113	0.019879248782121\\
1114	0.0156545331076131\\
1115	-0.00180538919843861\\
1116	-0.0176220655519637\\
1117	-0.0196583687677481\\
1118	0.0306614827648137\\
1119	0.0154364249081749\\
1120	0.011868797318619\\
1121	-0.00747822794955405\\
1122	-0.0452796214722678\\
1123	-0.168955210957747\\
1124	0.0220438499707258\\
1125	0.0137950274981261\\
1126	0.00969103622062302\\
1127	-0.0126308886946178\\
1128	-0.117671172672576\\
1129	0.102253530410265\\
1130	0.0200923112037639\\
1131	0.0129661590821665\\
1132	0.00809911553238937\\
1133	-0.0162390626140552\\
1134	-0.503923809009344\\
1135	0.0578370655107155\\
1136	0.0195051635311156\\
1137	0.0124978934735\\
1138	0.00688730192970326\\
1139	-0.0181502177780479\\
1140	0.706182761393132\\
1141	0.0466097012613266\\
1142	0.0193602189025209\\
1143	0.01222040559675\\
1144	0.00596433836646906\\
1145	-0.0188724250870286\\
1146	0.306495283319497\\
1147	0.0416964760272034\\
1148	0.0193708874153031\\
1149	0.0120502709960084\\
1150	0.00526104419123005\\
1151	-0.018975972158677\\
};
\addplot[forget plot, color=white!15!black] table[row sep=crcr] {%
0	0\\
1200	0\\
};
\addlegendentry{data1}

\end{axis}
\end{tikzpicture}%
	}
	\caption{Signal démodulé dans le receveur en essayant de compenser le
			déphasage de la porteuse en la filtrant également.}
\end{figure}

\section*{Conclusion}

\end{document}
