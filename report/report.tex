% !TEX TS-program = LuaLaTeX
\documentclass[10pt, oneside, a4paper]{article}

\usepackage[T1]{fontenc}
\usepackage{lmodern}
\usepackage{xcolor}
    \definecolor{gray} {HTML}{363636}
    \definecolor{red}  {HTML}{950009}
    \definecolor{green}{HTML}{0E610A}
    \definecolor{blue} {HTML}{020069}
    \definecolor{bglst}{HTML}{F4F4F4}
\usepackage{fontspec}
    \setsansfont{Arial}
\usepackage{amsmath}
\usepackage{titlesec}
    \titleformat*{\section}      {\color{gray}\large\bfseries\sffamily}
    \titleformat*{\subsection}   {\color{gray}\large\bfseries\sffamily}
    \titleformat*{\subsubsection}{\color{gray}\large\bfseries\sffamily}
\usepackage{geometry}
    \geometry{scale={0.75,0.85}}
\usepackage{siunitx}
    \sisetup{locale=FR}
\usepackage{graphicx}
\usepackage{caption}
    \captionsetup{labelfont={bf,sf,color=gray}}
\usepackage{pdfpages}
\usepackage{pgfplots}
	\pgfplotsset{compat=newest}
	\pgfplotsset{plot coordinates/math parser=false}
	\newlength\figureheight
	\newlength\figurewidth
\usepackage{minted}
	\setminted{linenos}
	\setminted{fontsize=\scriptsize}
	\setminted{bgcolor=bglst}

% Keep lasts
\usepackage[french]{babel}
	\frenchsetup{SmallCapsFigTabCaptions=false}
\usepackage[expansion]{microtype}
\usepackage[luatex, backref]{hyperref}
    \hypersetup{unicode, colorlinks, breaklinks, urlcolor=red,
                bookmarksopen, bookmarksnumbered}

% Generating plots is really time consuming
\usepgfplotslibrary{external}
	\tikzexternalize[prefix=figures/]

\renewcommand{\UrlFont}{\small}
\renewcommand{\arraystretch}{1.1}
\newcommand{\important}[1]{\textbf{\textsf{\color{gray}{#1}}}}
\setlength{\parskip}{8pt}

\begin{document}

\begin{titlepage}
    \centering
    \includegraphics[width=0.5\textwidth]{images/logo-ecam.png}\par
    \vspace{1cm}

    \rule{\linewidth}{1.5pt}%
    \vspace{5mm}
    {\rm\sffamily\LARGE Techniques de transmission et traitement du signal\par}
    \vspace{3mm}
    {\sffamily\bfseries\LARGE Simulation d’une chaîne de transmission\\
    						  numérique avec Matlab\textregistered{}\par}
    \vspace{5mm}
    \rule{\linewidth}{1.5pt}%
    \vspace{1cm}

    {\large%
        \begin{minipage}[t]{0.35\linewidth}
            \centering
            Alexis~\bsc{Nootens} \\[1mm]
            \href{mailto:16139@student.ecam.be}{16139@student.ecam.be}
        \end{minipage}
        \begin{minipage}[t]{0.35\linewidth}
            \centering
            Armen~\bsc{Hagopian} \\[1mm]
            \href{mailto:14040@student.ecam.be}{14040@student.ecam.be}
        \end{minipage}
    \par}
    \vspace{1cm}

    {\large%
        ECAM Brussels             \\[1mm]
        Promenade de l'Alma 50    \\[1mm]
        1200 Woluwe-Saint-Lambert \\[1mm]
        Belgique
    \par}

    \vfill
    {\large\today\par}
\end{titlepage}

%%%%%%%%%%%%%%%%
\tableofcontents
\newpage

%%%%%%%%%%%%%%%%%%%%%%%
\section{Introduction}
L'objectif de ce projet est simuler la couche physique d'un protocol de communication, c'est-à-dire le niveau 1 du modèle OSI.
La simulation est réalisée avec le logiciel Matlab\textregistered{} édité par Mathworks\textregistered{}.
Les contraintes imposées dans la simulation sont de tenir compte de plusieurs émetteurs et receveurs pouvant communiquer en même temps.
Pour répondre à cette contrainte, la couche physique implémentée utilise le multiplexage fréquentiel.

Ce document reprend la conception du projet et les choix qui ont dû y être décidés, accompagnés de leur explication.

\section{L'émetteur}

\section{Le canal}

\section{Le receveur}

\section{Les performances}

\section{Conclusion}

\newpage
\appendix
\section{main.m}
\inputminted{matlab}{../main.m}

\pagebreak
\section{parameters.m}
\inputminted{matlab}{../parameters.m}

\pagebreak
\section{sender.m}
\inputminted{matlab}{../sender.m}

\pagebreak
\section{channel.m}
\inputminted{matlab}{../channel.m}

\pagebreak
\section{receiver.m}
\inputminted{matlab}{../receiver.m}


\end{document}
